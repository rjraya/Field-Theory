\begin{exercise}
Sea $\frac{E}{K}$ una extensión finita de Galois y $\alpha \in \frac{E}{K}$. Vamos a determinar el polinomio irreducible de $\alpha$ sobre $K$. 

Consideramos todos los conjugados de $\alpha$, esto es, el conjunto $C = \{\sigma(\alpha):\sigma \in Gal(\frac{E}{K}) \}$ en el que no hay elementos repetidos ya que es un conjunto, y definimos $f(X) = \prod_{\beta \in C} (X - \beta)$. Entonces $\alpha$ es una raíz de $f(X)$. 

\begin{enumerate}
\item Prueba que $f(X) \in K[X]$
\item Prueba que $f(X)$ es irreducible sobre $K$. Como consecuencia $f(X) = Irr(\alpha,K)$.
\item Considera la extensión $\frac{\mathbb{Q}(\sqrt[3]{3},\sqrt{-3})}{\mathbb{Q}}$, prueba que es una extensión de Galois. 
\item Determinar  $Gal \Big(\frac{\mathbb{Q}(\sqrt[3]{3},\sqrt{-3})}{\mathbb{Q}}\Big)$ y $Irr(\sqrt[3]{3}+\sqrt{-3},\mathbb{Q})$
\item Considera la clausura normal $E/\mathbb{Q}$ de $\mathbb{Q}(\sqrt[4]{2})/\mathbb{Q}$, que es una extensión de Galois y describe $E$. 
\item Determina $Gal(E/\mathbb{Q})$ e $Irr(i + \sqrt{2} +\sqrt{-2},\mathbb{Q})$.
\end{enumerate}
\end{exercise}

\textbf{Solución:}

Discutamos primero la información que proporciona el enunciado. Teníamos la definición de conjugados sobre un cuerpo $K$ que podía ser caracterizada como aquellos elementos de la clausura $u,v \in \overline{K}$ que tenían el mismo polinomio mínimo sobre el cuerpo, $Irr(u,K) = Irr(v,K)$.

Desde aquí vamos a observar que dado un $\alpha \in E \setminus K$, sus elementos conjugados son precisamente los que dice el enunciado (si $\alpha \in K$ él es su único conjugado). 

Observamos que los automorfismos $\sigma \in Gal(\frac{E}{K})$ extienden a la identidad en $K$ pues por definición fijan los elementos de $K$. Como el isomorfismo extendido a polinomios con $X \to X$ de la identidad es $\overline{1_K} = 1_{K[X]}$ en particular se tiene que $Irr(\alpha,K) = Irr(\sigma(\alpha),K)$. Por tanto, $C$ está incluido en los conjugados de $\alpha$. 

Pero también se da el recíproco, esto es los conjugados de $\alpha$ están todos en $E$. En efecto, como la extensión es de Galois finita entonces en particular es normal y entonces $Irr(\alpha,K)$ al ser irreducible y tener una raíz en $E$ descompone en factores lineales en $E$ luego cualquier conjugado de $\alpha$ también está en $E$. 

Finalmente, como $Gal(\frac{E}{K})$ es un grupo claramente $\alpha \in C$ con $\sigma = 1$. 

Volviendo al ejercicio:

\begin{enumerate}
\item Sea $\sigma \in Gal(\frac{E}{K})$ sabemos que como $\sigma$ extiende a la identidad las raíces del polinomio $f$ se aplican en raíces del polinomio $f$ mediante $\sigma$ y como $\sigma$ es un automorfismo, determina sobre las raíces una permutación. 

Entonces cuando consideramos la extensión de $\sigma$ a los polinomios con $X \to X$ que denotamos por $\overline{\sigma}$ lo que obtenemos es $$\overline{\sigma}(f)(x) = \overline{\sigma}(\prod (x - \beta)) = \prod (x - \sigma(\beta)) = f(x)$$ Por tanto, $\sigma$ deja invariantes a los coeficientes del polinomio y como esta igualdad no depende de $\sigma$ se tiene que $f \in E^G[X] = K[X]$
\item Por la observación previa tenemos que $C$ y el conjunto de los conjugados de $\alpha$ coinciden y en particular tienen el mismo cardinal. 

Observamos también que como la extensión es de Galois finita, es una extensión separable, esto quiere decir que el polinomio $Irr(\alpha,K)$ no tiene raíces múltiples en $E$, cuerpo en el que descompone totalmente. Por tanto, cada conjugado determina un único factor lineal esto es, $$gr(f) = |\{ \text{conjugados de  } \alpha \}|= gr(Irr(\alpha,K))$$ y dado que $\alpha$ es raíz de $f$ tenemos que $f$ es un polinomio de grado mínimo del que $\alpha$ es raíz. Por tanto, $f$ es irreducible sobre $K$ y $f = Irr(\alpha,K)$. 

\item La extensión es de Galois puesto que es el cuerpo de descomposición del polinomio: $$(X^3-3)(X^2+3) \in \mathbb{Q}[X]$$ que es separable por ser $\mathbb{Q}$ un cuerpo de característica cero y por tanto perfecto. 

\item Consideramos la torre: $$\mathbb{Q} \subseteq \mathbb{Q}(\sqrt[3]{3}) \subseteq \mathbb{Q}(\sqrt[3]{3},\sqrt{-3})$$ El polinomio $X^3-3$ es irreducible sobre $\mathbb{Q}$ ya que no tiene raíces en $\mathbb{Q}$ y entonces la primera extensión es de grado 3. La base de la segunda extensión está incluída en $\mathbb{R}$ y entonces claramente el polinomio $X^2 + 3$ es irreducible sobre $\mathbb{Q}(\sqrt[3]{3})$ y por tanto la segunda extensión es de grado 2. Por el teorema del grado, la extensión total es de grado 6. Por tanto, hay exactamente 6 elementos en el grupo de Galois de la extensión. Como grupos de orden $2p$ con $p$ primo sólo hay un cíclico y un diédrico. Por tanto, para clasificar este grupo estudiamos el orden de sus elementos. 

Para hacer los cálculos hay que observar que $\omega = -\frac{1}{2} + \frac{\sqrt{-3}}{2}$ y que $\omega^2 = -\frac{1}{2} - \frac{\sqrt{-3}}{2}$.

\begin{center}
  \begin{tabular}{ l | c | c | r }
    \hline
    notación & $\sqrt[3]{3}$ & $\sqrt{-3}$ & orden \\ \hline
    $Id$ & $\sqrt[3]{3}$ & $\sqrt{-3}$ & 1 \\ \hline
    $s$ & $\sqrt[3]{3}$ & $-\sqrt{-3}$ & 2 \\ \hline
    $r$ & $\omega \sqrt[3]{3}$ & $\sqrt{-3}$ & 3 \\ \hline
    $rs$ & $\omega \sqrt[3]{3}$ & $-\sqrt{-3}$ & 2 \\ \hline
    $r^2$ & $\omega^2 \sqrt[3]{3}$ & $\sqrt{-3}$ & 3 \\ \hline
    $r^2s$ & $\omega^2 \sqrt[3]{3}$ & $-\sqrt{-3}$ & 2 \\
    \hline
  \end{tabular}
\end{center}

En estas condiciones es isomorfo a $D_3$.

Para calcular $Irr(\sqrt[3]{3}+\sqrt{-3},\mathbb{Q})$ tenemos que hacer algunos cálculos:

$$\alpha = \sqrt[3]{3}+\sqrt{-3} \implies (\alpha - \sqrt{-3})^3 = 3 \implies \alpha^3-3\alpha-3 = \sqrt{-3}(\alpha^2-3) \implies \alpha^6 - 6 \alpha^4-6\alpha^3+3\alpha^2+18\alpha+9 = 0$$

Para investigar la naturaleza de este polinomio lo reducimos módulo $2$ obteniendo $p(X) = X^6+X^2+1$. Vemos que no tiene raíces y entonces no puede tener en $\mathbb{Z}[X]$ factores de grado 1 o 5. El único irreducible de grado en $\mathbb{Z}_2[X]$ es $X^2+X+1$ y comprobamos que el resto de la división euclídea es $X+1$ y por tanto, el polinomio no tiene factores de grado 2 o 4. Sin embargo resulta que $p(X) = (X^3+X+1)^2$ lo que nos dice que si tuviera factores serían de grado 3. 

Volviendo a $\mathbb{Z}$ vemos por ensayo y error que $X^6 - 6 X^4-6X^3+3X^2+18X+9 = (X^3-3X-3)^2$ y por el criterio de Eisenstein con $p = 3$ se obtiene que $X^3-3X-3$ es irreducible. De donde, $Irr(\sqrt[3]{3}+\sqrt{-3},\mathbb{Q}) = X^3-3X-3$. 

\item  Claramente, $Irr(\sqrt[4]{2},\mathbb{Q}) = X^4 - 2$ y su cuerpo de descomposición es $\mathbb{Q}(\sqrt[4]{2},i)/\mathbb{Q}$. Esta es la clausura normal ya que la extensión es finita y tiene un solo generador. La clausura normal siempre es una extensión normal y en este caso vemos claramente que la extensión es separable puesto que los generadores tienen polinomios mínimos sobre $\mathbb{Q}$ son $X^4-2,X^2+1$ que tienen todas sus raíces distintas, luego son separables. Por tanto, tenemos una extensión de Galois. Vamos a describir esta extensión. 

Considerando la torre $\mathbb{Q} \subseteq \mathbb{Q}(\sqrt[4]{2}) \subseteq \mathbb{Q}(\sqrt[4]{2},i)$ vemos claramente que la extensión tiene grado 8 y por el teorema de Artin, $|Gal(\mathbb{Q}(\sqrt[4]{2},i)/\mathbb{Q})| = 8$. Para determinar el grupo calculamos los de los elementos.

\begin{center}
  \begin{tabular}{ l | c | c | r }
    \hline
    notación & $i$ & $\sqrt[4]{2}$ & orden \\ \hline
    $Id$ & $i$ & $\sqrt[4]{2}$ & 1 \\ \hline
    $\sigma$ & $i$ & $i\sqrt[4]{2}$ & 4 \\ \hline
    $\sigma^2$ & $i$ & $-\sqrt[4]{2}$ & 2 \\ \hline
    $\sigma^3$ & $i$ & $-i\sqrt[4]{2}$ & 4 \\ \hline
    $\tau$ & $-i$ & $\sqrt[4]{2}$ & 2 \\ \hline
    $\sigma\tau$ & $-i$ & $-\sqrt[4]{2}$ & 2 \\ \hline
    $\sigma^2\tau$ & $-i$ & $i\sqrt[4]{2}$ & 2 \\ \hline
    $\sigma^3\tau$ & $-i$ & $-i\sqrt[4]{2}$ & 2 \\ 
    \hline
  \end{tabular}
\end{center}

Si observamos que el grupo no es abeliano entonces tenemos que el grupo es $D_4$. 

La correspondencia de Galois viene expresada mediante los siguientes diagramas:

\begin{tikzcd}
& & \{e\} \arrow{drr} \arrow{dr} \arrow{d}  \arrow{dl} \arrow{dll} \\
\langle \tau \rangle  \arrow{dr} &
\langle \sigma^2\tau \rangle \arrow{d}  &
\langle \sigma^2 \rangle \arrow{dr} \arrow{dl} \arrow{d} \arrow{dr} &
\langle \sigma\tau \rangle \arrow{d} &
\langle \sigma^3\tau \rangle  \arrow{dl} \\
& \langle \sigma^2,\tau \rangle  \arrow{dr}  &
\langle \sigma \rangle \arrow{d} &
\langle \sigma^2,\sigma \tau \rangle \arrow{dl}  \\
& & Gal(\frac{\mathbb{Q}(\sqrt[4]{2},i)}{\mathbb{Q}})
\end{tikzcd}


Los subgrupos de orden 2 son cíclicos por ser 2 un primo y por tanto, están generados por los elementos del grupo de orden 2 y por tanto hay 5. Los subgrupos de orden $4 = 2^2$ son o bien cíclicos o bien producto de cíclicos de orden 2. Estos últimos son los llamados de tipo Klein y para identificarlos debemos tomar dos elementos de orden 2 que conmuten entre sí. Se obtienen los subgrupos representados en la figura superior. 

Observamos también que todos los de orden 4, por ser de índice 2 son subgrupos normales y entre los de orden 2 el único que es normal es el generado por el elemento que está el centro del grupo $\sigma^2$. 

Ahora pasamos a determinar la conexión de Galois entre estos subgrupos y los subcuerpos de la extensión $\mathbb{Q}(\sqrt[4]{2},i)/\mathbb{Q})$. Para ello determinamos los cuerpos fijos por cada subgrupo ya que están en biyección. Para elo vemos como actúa cada isomorfismo sobre los elementos de una $\mathbb{Q}$-base.

\begin{center}
  \begin{tabular}{ l | c | c | c | c | c | c | r }
    \hline
    &  $\alpha$ & $\alpha^2$ & $\alpha^3$ & $i$ & $i\alpha$ & $i\alpha^2$ & $i\alpha^3$ \\ \hline
    $\sigma$ & $i\alpha$ & $-\alpha^2$ & $-i\alpha^3$ & $i$ & $-\alpha$ & $-i\alpha^2$ & $\alpha^3$ \\ \hline 
    $\sigma^2$ & $-\alpha$ & $\alpha^2$ & $-\alpha^3$ & $i$ & $-i\alpha$ & $i\alpha^2$ & $-i\alpha^3$ \\ \hline
    $\sigma^3$ & $-i\alpha$ & $-\alpha^2$ & $i\alpha^3$ & $i$ & $\alpha$ & $-i\alpha^2$&  $-\alpha^3$ \\ \hline
    $\tau$ & $\alpha$ & $\alpha^2$ & $\alpha^3$ & $-i$ & $-i\alpha$ & $-i\alpha^2$ & $-i\alpha^3$ \\ \hline
    $\sigma \tau$ & $i\alpha$ & $-\alpha^2$ & $-i\alpha^3$ &  $-i$ & $\alpha$ & $i\alpha^2$ & $-\alpha^3$ \\ \hline
    $\sigma^2\tau$  & $i\alpha$ & $-\alpha^2$ & $-i\alpha^3$ & $-i$ & $\alpha$ & $i\alpha^2$ & $-\alpha^3$ \\ \hline
    $\sigma^3\tau$  & $-i\alpha$ & $-\alpha^2$ & $-i\alpha^3$ & $-i$ & $-\alpha$ & $i\alpha^2$ & $-i\alpha^3$ \\ 
    \hline
  \end{tabular}
\end{center}

Vamos a ir determinando los cuerpos fijos sencillos.

$E^G = \mathbb{Q}$

$E^{\sigma} = \mathbb{Q}(i)$

$E^{\sigma^2,\sigma\tau} = E^{\sigma^2} \cap E^{\tau} = \mathbb{Q}(\sqrt{2})$

$E^{\sigma^2,\sigma\tau} = E^{\sigma^2} \cap E^{\sigma\tau} = \mathbb{Q}(i\sqrt{2})$

$E^{\sigma^2} = E^{\sigma^2,\sigma\tau}E^{\sigma^2,\tau} = \mathbb{Q}(i,\sqrt{2})$

$E^{\tau} = \mathbb{Q}(\sqrt[4]{2})$

Cuando llegamos al $\sigma\tau$ obtendríamos que fija $i\alpha^2$ pero ya había un cuerpo fijo de esta forma. De hecho, ya que $\sigma\tau$ tiene orden 2, el teorema de Artin asegura que $[E:E^H]=2$ pero como $i\alpha^2$ determina una extensión de grado 2 sobre $\mathbb{Q}$, en realidad el subgrupo debería tener 4 elementos lo cual no es posible. Tenemos que rellenar elementos hasta que el grado de la extensión sobre $\mathbb{Q}$ sea $4$. 

Para hallar los elementos fijos planteamos la ecuación para $\sigma^3\tau(u) = u$ y escribimos $$u = a_0 \cdot 1 + a_1 \alpha + a_2 \alpha^2 +a_3 \alpha^3 + a_4i + a_5i\alpha + a_6i\alpha^2+a_7i\alpha^3$$ Operando se llega a que todos los coeficientes pueden ser elegidos como $0$ y que $a_1 = -a_5$. Haciendo $a_1 = -a_5$ obtenemos un elemento $u = \alpha - i \alpha = (1-i) \alpha$. El proceso análogo para $\sigma\tau$ da $u = (1+i)\alpha$. Por tanto los cuerpos correspondientes son:

$E^{\sigma\tau} = \mathbb{Q}((1+i)\alpha)$

$E^{\sigma^3\tau} = \mathbb{Q}((1-i)\alpha)$

Tenemos entonces el siguiente diagrama de cuerpos:

\begin{tikzcd}
& & \mathbb{Q}(i,\sqrt[4]{2}) \\
\mathbb{Q}(\sqrt[4]{2}) \arrow{urr} &
\mathbb{Q}(i\sqrt[4]{2}) \arrow{ur} &
\mathbb{Q}(i,\sqrt{2}) \arrow{u} &
\mathbb{Q}((i+1)\sqrt[4]{2}) \arrow{ul} &
\mathbb{Q}((1-i)\sqrt[4]{2}) \arrow{ull} \\
& \mathbb{Q}(\sqrt{2}) \arrow{ul} \arrow{u} \arrow{ur} &
\mathbb{Q}(i) \arrow{u} &
\mathbb{Q}(i\sqrt{2}) \arrow{ul} \arrow{u} \arrow{ur} \\
& & \mathbb{Q} \arrow{ul} \arrow{u} \arrow{ur}
\end{tikzcd}

\item Por cálculo directo, vemos que si $\alpha = i + \sqrt{2} + \sqrt{-2}$ entonces sería raíz de $p(X) = X^4 + 2 X^2 + 16 X + 17$. Comprobamos que es difícil reduciendo ver que el polinomio es irreducible. En vez ello observamos que $\alpha$ está en la extensión $\mathbb{Q}(\sqrt{2},i)$ y no está en ninguna de las subextensiones. Por tanto, $p(X) = Irr(\alpha,\mathbb{Q})$. 
\end{enumerate}