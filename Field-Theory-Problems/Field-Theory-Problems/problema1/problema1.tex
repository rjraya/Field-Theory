\begin{lemma}\label{numeros-algebraicos}
Dada cualquier extensión de cuerpos $\frac{F}{K}$, el conjunto $$M = \{\alpha \in F:\alpha \text{ es algebraico sobre } K\}$$ es un subcuerpo de $F$ que contiene a $K$. Además la extensión $\frac{M}{K}$ es algebraica. 
\end{lemma}
\begin{proof}
Veamos que $M$ es cerrado para sumas y productos. En efecto, si $\alpha,\beta \in M$ entonces $\frac{K(\alpha,\beta)}{K}$ es una extensión finita ya que es de generación finita por generadores algebraicos. Como es una extensión finita, tiene que ser algebraica. 

Entonces todo elemento de $K(\alpha,\beta)$ es algebraico sobre $K$ y claramente $\alpha + \beta, \alpha \beta \in K(\alpha,\beta)$, luego son algebraicos sobre $K$.

Por otro lado, es claro que $K \subseteq M$ ya que si $\alpha \in K$ entonces ciertamente, $\alpha$ es raíz del polinomio $X-\alpha$. En particular, $1,-1 \in M$ y por tanto, se tiene que $M$ es un subanillo. Nos falta comprobar que también es cerrado para inversos de los elementos no nulos. 

Sea $\alpha \neq 0$ algebraico, entonces $\alpha$ es raíz de algún polinomio con coeficientes en $K$. Sea este $p = \sum_{i = 0}^n a_i X^i$ entonces es claro que $\sum_{i = 0}^{n} a_{n-i}X^i$, si evaluamos en $\frac{1}{\alpha}$ y multiplicamos por $\alpha^n$ es claro que $\frac{1}{\alpha}$ es raíz de este polinomio. En resumen $\frac{1}{\alpha} \in M$. 

Por tanto, $M$ es un cuerpo que contiene a $K$. Claramente, la extensión $\frac{M}{K}$ es algebraica ya que todos los elementos de $\alpha$ son raíces de alǵun polinomio con coeficientes en $K$. 
\end{proof}

\begin{lemma}\label{conjugado}
Sea $\frac{F}{K}$ una extensión de cuerpos, $p \in K[x]$ y $z \in F$ tal que $p(z) = 0$. Si $\sigma:F \to F$ es un endomorfismo de anillos tal que $\sigma|_{K} = Id|_{K}$ entonces $\sigma(z)$ es raíz de $p$. 
\end{lemma}
\begin{proof}
Sea $p(X) = \sum a_i X^i$ entonces $p(\sigma(z)) = \sum a_i \sigma(z)^i$ y ya que $\sigma$ fija los elementos de $K$ lo anterior es igual a $\sigma(p(z))$ como $p(z) = 0$ por hipótesis, se tiene que $p(\sigma(z)) = 0$. 
\end{proof}

\begin{exercise}
Se considera $F_1$ (resp. $F_2$) el subcuerpo de $\mathbb{R}$ (resp. $\mathbb{C}$) de todos los elementos algebraicos sobre $\mathbb{Q}$. Probar que $\frac{F_i}{\mathbb{Q}}$ es una extensión algebraica y que $[F_2:F_1] = 2$. 
\end{exercise}
\begin{proof}
Basta tomar $F = \mathbb{R},\mathbb{C}$ y $K = \mathbb{Q}$ en el lema \ref{numeros-algebraicos}

Por otro lado, se tiene la siguiente cadena de cuerpos $\mathbb{Q} \subseteq F_1 \subseteq F_2 \subseteq \mathbb{C}$. Vamos a ver que de hecho $F_2 = F_1[i] \cong \frac{F_1[X]}{<X^2+1>}$ de donde la dimensión es claramente dos, esto es, $[F_2:F_1] = 2$. 

$F_1[i] \subseteq) F_2$ Claramente, $1,i \in F_2$ ya que son números complejos y son raíces de los polinomios $X-1,X^2-1 \in \mathbb{Q}[X]$. También se verifica que $F_1 \subseteq F_2$. Como $F_2$ es un cuerpo, cualquier combinación lineal de sus elementos $\alpha + \beta i$ con $\alpha,\beta \in F_1$ está en $F_2$. Esto completa la inclusión. 

$F_2 \subseteq F_1[i])$ Sea $\alpha + \beta i \in F_2$ raíces de cierto polinomio $p \in \mathbb{Q}[X]$ y queremos ver que $\alpha,\beta \in F_1$. Por el lema \ref{conjugado}, ya que $\sigma(\alpha + i\beta ) = \alpha - i \beta$ es un endomorfismo por cálculo directo, se tiene que la raíz conjugada $\alpha - i \beta$ también es raíz de $p \in \mathbb{Q}[X]$.  En consecuencia como claramente $\frac{1}{2}, \frac{1}{2i} \in F_2$ y $F_2$ es cuerpo y como $\alpha = \frac{\alpha + i \beta}{2} + \frac{\alpha -i \beta}{2} \land \beta = \frac{\alpha + i\beta}{2i} - \frac{\alpha - i\beta}{2i}$ se tiene que $\alpha,\beta \in F_2$ y como $\alpha,\beta \in \mathbb{R}$ se verifica claramente que $\alpha,\beta \in F_1$, como se quería.  
\end{proof}

\pagebreak

\begin{exercise}
Calcular:

\begin{itemize}
\item $Irr(\sqrt{2},\mathbb{F}_3)$
\item $Irr(\sqrt[4]{2},\mathbb{F}_3)$
\item $Irr(\sqrt{2}+\sqrt[4]{2},\mathbb{F}_3)$
\item $Irr(\sqrt{2}+\sqrt[4]{2},\mathbb{Q})$
\end{itemize}
\end{exercise}

Para determinar los valores $\alpha$ que corresponden a cada una de las expresiones ambiguas $\sqrt[n]{}$ nos vamos a ir a un cuerpo extensión. Uno natural es $ \mathbb{F}_9 \cong \frac{\mathbb{F}_3[X]}{<x^2+2x+2>} \cong \mathbb{F}_3[\sqrt{2}]$.

1. Por definición si $\alpha = \sqrt{2}$ entonces $\alpha^2 - 2 = 0$ y por tanto, $\alpha$ será un número algebraico sobre $\mathbb{F}_3$ que anula al polinomio $p(X) = X^2-2$. Este polinomio es irreducible y mónico sobre $\mathbb{F}_3$ ya que no tiene raíces en $\mathbb{F}_3$.  Como $\alpha$ es una raíz suya deducimos que $Irr(\alpha,\mathbb{F}_3) = X^2-2$.

Obsérvese que en el cuerpo extensión estas raíces son $\alpha = \sqrt{2},-\sqrt{2}$. 

2.Por definición si $\alpha = \sqrt[4]{2}$ entonces $\alpha^4 - 2 = 0$ y por tanto, $\alpha$ será un número algebraico sobre $\mathbb{F}_3$ que anula al polinomio $p(X) = X^4-2$. Este polinomio no es irreducible sobre $F_3$. Veámoslo.

Como $p$ no tiene raíces sobre $\mathbb{F}_3$, no tiene factores de grado 1 ni de grado 3. Luego sólo puede tener factores de grado 2. Los irreducibles de grado 2 en $\mathbb{F}_3[X]$ son $X^2+1,X^2+X+2,X^2+2X+2$ y realizando la división euclídea se tiene que $p(X) = (X^2+X+2)(X^2+2X+2)$. Por ser $\alpha$ una raíz del polinomio $p$ será una raíz del polinomio $X^2+X+2$ o $X^2+2X+2$ que como son mónicos e irreducibles son candidatos a ser $Irr(\alpha,\mathbb{F}_3)$.

 Las raíces $\alpha = 2+\sqrt{2},2-\sqrt{2}$ tienen $Irr(\alpha,\mathbb{F}_3) = X^2+2X+2$ y las raíces $\alpha = 1 + 2\sqrt{2}, 1-2\sqrt{2}$ tienen $Irr(\alpha,\mathbb{F}_3) = X^2+X+2$.


3. Primero obtenemos un polinomio que se anule en $\alpha$ mediante cálculo directo: $$\alpha = \sqrt{2}+\sqrt[4]{2}$$ $$(\alpha - \sqrt{2})^2 = \sqrt{2}$$ $$(\alpha^2+2)^2 = ((2 \alpha + 1) \sqrt{2})^2$$ $$\alpha^4 - 4 \alpha^2 - 8 \alpha + 2 = 0$$ $$\alpha^4 - \alpha^2 + \alpha + 2 = 0$$ $$(\alpha-1)^2(\alpha^2+2\alpha^2+2)$$

La raíz $\alpha = 1$ tiene $Irr(\alpha,\mathbb{F}_3) = X-1$ y las raíces $\alpha = 2+\sqrt{2},2+2\sqrt{2}$ tienen $Irr(\alpha,\mathbb{F}_3) = X^2+2X+2$. 

Nos damos cuenta que esto no puede contener a todas los sumandos de las raíces halladas en los apartados 1 y 2. La razón es que en el primer paso hemos asumido implícitamente que $\sqrt{2} = (\sqrt[4]{2})^2$. Pero también podríamos elegir $\sqrt{2} = - (\sqrt[4]{2})^2$. Los cálculos en este caso son  los siguientes:

$$\alpha = \sqrt{2}+\sqrt[4]{2}$$ $$(\alpha - \sqrt{2})^2 = -\sqrt{2}$$ $$(\alpha^2+2)^2 = ((2 \alpha - 1) \sqrt{2})^2$$ $$\alpha^4 - 4 \alpha^2 + 8 \alpha + 2 = 0$$ $$\alpha^4 - \alpha^2 - \alpha + 2 = 0$$ $$(\alpha-2)^2(\alpha^2+\alpha+2)$$

Por tanto, si $\alpha = 2$ obtengo que $Irr(\alpha,\mathbb{F}_3) = X-2$ y si $\alpha = 1+\sqrt{2},1-\sqrt{2}$ tienen $Irr(\alpha,\mathbb{F}_3) = X^2+X+2$.

4. Repitiendo el proceso anterior,  llegamos al polinomio $$\alpha^4 - 4 \alpha^2 - 8 \alpha + 2 = 0$$ Esto nos da como candidato a polinomio mínimo $p(X) = X^4 - 4X^2 - 8 X + 2$. 

Sabemos que estudiar la irreducibilidad  de un polinomio primitivo en $\mathbb{Z}$ es equivalente a estudiarla en $\mathbb{Q}$. Entonces en nuestro caso basta aplicar el criterio de Eisenstein con un primo $p = 2$ para obtener que es irreducible. 

Dado que $p$ es irreducible, mónico y $\alpha$ debe ser raíz de $p$ deducimos que $$Irr(\sqrt{2}+\sqrt[4]{2},\mathbb{Q}) = X^4 - 4X^2 - 8 X + 2$$