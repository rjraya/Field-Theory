\begin{definition}[Extensiones cíclicas y abelianas]
Una extensión $E/K$ es cíclica si es de Galois y su grupo de Galois es cíclico.

Una extensión $E/K$ es abeliana si su grupo de Galois es un grupo abeliano.  
\end{definition}

\subsection{Teorema 90 de Hilbert}

Los siguientes dos teoremas se pueden entender como descriptores del núcleo de los homomorfismos norma y traza definidos en el apartado anterior, en el caso de extensiones cíclicas. 

\begin{theorem}[Teorema 90 de Hilbert]
Sea $E/K$ extensión cíclica de grado $n$ con grupo $G = Gal(E/K) = \langle \sigma\rangle$ y
sea $\beta \in E$. Son equivalentes:

\begin{enumerate}
\item $N_{E/K}(\beta) = 1$
\item Existe $0 \neq \alpha \in E$ tal que $\beta = \frac{\alpha}{\sigma(\alpha)}$.
\end{enumerate}
\end{theorem}
\begin{proof}
$\Leftarrow)$ Usando que la norma es homomorfismo multiplicativo, se tiene que: \[
N_{E/K}(\beta) = \frac{N_{E/K}(\alpha)}{N_{E/K}(\sigma(\alpha))} = 1
\]

$\Rightarrow)$ Si tomamos $\beta$ con $N(\beta) = 1$ esto implica que $\beta \neq 0$. Además, los automorfismos $1,\sigma,\dots,\sigma^{n-1}$ son distintos dos a dos ya que si $\sigma^i = \sigma^j$ con $i > j$ entonces se deduciría que el orden del grupo de Galois es menor que $n$ y entonces por el teorema de Artin, el grado de la extensión no sería $n$. 

Como los automorfismos son todos distintos, por el lema de Dedekind son linealmente independientes como homomorfismos de espacios vectoriales sobre $K$. En particular, la aplicación: \[
\tau 
= 
1 + \beta\sigma + (\beta \sigma(\beta))\sigma^2 + 
\dots +
(\beta \sigma(\beta)\dots \sigma^{n-2}\beta) \sigma^{n-1}
\]

no es idénticamente cero. En particular, $\exists \theta \in E.\tau(\theta) \neq \theta$. Sea $\alpha = \tau(\theta)$. Calculamos $\sigma(\alpha)$ en la expresión anterior y multiplicamos por $\beta$. Teniendo en cuenta que $\sigma^n = 1$ y $\beta \sigma(\beta) \ldots \sigma^{n-1}(\beta))\sigma^n(\beta) = 1$ quedaría $\beta \sigma(\alpha) = \alpha$. 
\end{proof}

\begin{theorem}[Teorema 90 de Hilbert aditivo]
Sea $K$ un cuerpo y $E/K$ una extensión cíclica de grado $n$ con grupo $G = Gal(E/K) = \langle \sigma \rangle$ y sea $\beta \in E$. Son equivalentes: 

\begin{enumerate}
\item $T_{E/K}(\beta) = 0$.
\item $\exists \alpha \in E. \beta = \alpha - \sigma(\alpha)$. 
\end{enumerate}
\end{theorem}

\subsection{Clasificación de los grupos cíclicos}

\begin{theorem}[Teorema de Lagrange]
Sea $K$ cuerpo y $n \in \mathbb{N} \setminus \{0\}$. 

Si $p = Car(K) \neq 0$ supondremos además que $p \nmid n$. 

Supongamos que existe una raíz n-ésima primitiva de la unidad en $K$.

\begin{enumerate}
\item Si $E/K$ es una extensión cíclica de grado $n$, entonces existe $\alpha \in E$ tal que: $$E = K(\alpha) \text{ y } Irr(\alpha,K) = X^n-a \text{ para algún } a \in K$$
\item Sea $a \in K$. Si $\alpha$ es una raíz de $X^n-a$, entonces se tiene $K(\alpha)/K$ es una extensión cíclica de grado $d$ con $d \mid n$ y $\alpha^d \in K$.
\end{enumerate}
\end{theorem}

La lectura adecuada del teorema anterior sería que considerando cuerpos de característica nula o prima $p$ donde $p$ no divide a cierto $n$ que representa un grado de extensión o un índice de raíz, se tiene que toda extensión cíclica de grado $n$ se obtiene como una extensión simple generada por una raíz n-ésima. Recíprocamente, partiendo de una raíz n-ésima genero una extensión cíclica de grado divisor de $n$ tal que cierta potencia se puede ver en $K$ (y no en $K(\alpha)$).

Un ejemplo sencillo de lo anterior nos lo da $X^2+1$ cuyo grupo de Galois consiste de la conjugación compleja y de la identidad. 

\begin{theorem}[Teorema de Artin-Schreier]
Sea $K$ un cuerpo de característica $p$. Se verifica:

\begin{enumerate}
\item Si $E/K$ es una extensión cíclica de grado $p$, entonces existe $\alpha \in E$ tal que: $$E = K(\alpha) \text{ y } Irr(\alpha,K) = X^p-X-a \text{ para algún } a \in K$$ 
\item Sea $a \in K$. Tenemos dos casos según la estructura de $X^p-X-a$:

\begin{enumerate}
\item Tiene una raíz en $K$. En este caso, todas las raíces están en $K$ (no da extensión cíclica).
\item Es irreducible. En este caso, si $\alpha$ es una raíz, la extensión $K(\alpha)/K$ es cíclica de grado $p$.
\end{enumerate}
\end{enumerate}
\end{theorem}

Este teorema completaría una caracterización de las extensiones cíclicas. Por un lado, se tiene el caso en que $p \nmid n$. Por otro lado, el caso en que $p|n$. Este caso se ha disgregado implícitamente en dos subcasos. El caso $n = p$ que es el caso anterior y el caso $n = p^mt$ con $m \ge 1$. Este caso, se puede reducir al caso anterior.

En efecto, como el grupo de Galois de la extensión es cíclico tenemos para cada divisor un subgrupo de ese orden. Entonces, haciendo la correspondencia entre grupos y cuerpos obtendríamos un diagrama similar al siguiente:

\begin{tikzcd}
E_n \arrow[dash]{d}{p} & \\
E_{n-1}  \arrow[dash]{d}{p} & \\
\ldots \arrow[dash]{d}{p} & \\
E_{1}  \arrow[dash]{d}{t} & \\
K
\end{tikzcd}

donde todas las extensiones que nacen son cíclicas y se pueden estudiar mediante los criterios anteriores. 









