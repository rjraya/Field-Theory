\subsection{Resultados generales}

\begin{theorem}[Caracterización de los cuerpos finitos]
1. Si K es un cuerpo finito, entonces su $|K| = p^n$ con $p$ un número primo que es la característica del cuerpo y $n \ge 1$. 

2. Teorema de Moore: para cada primo $p$ y cada $n \ge 1$ existe un único cuerpo de orden $p^n$ salvo isomorfismo, este es, el cuerpo de descomposición del polinomio $X^{p^n} - X$ sobre $\mathbb{F}_p$. 
\end{theorem}
\begin{proof}
Si $K$ es finito, necesariamente su característica ha de ser un número primo y por tanto por el primer teorema de isomorfía $Img(f) \cong \mathbb{Z}_p$ es un subcuerpo de $K$. Subcuerpo que denotaremos por $\mathbb{F}_p$.

Ahora, $K$ tiene estructura de espacio vectorial sobre $\mathbb{F}_p$ y dado que $K$ es finito, considerando todos los elementos de $K$ como vectores del espacio vectorial sobre $\mathbb{F}_p$, claramente, $K$ es finitamente generado por sus elementos. Sabemos del álgebra lineal que todo espacio vectorial finitamente generado tiene una base. El cardinal de esta base nos dará la dimensión $n$ y todo vector de $K$ se expresa de forma en función de $n$ vectores. Por tanto, hay exactamente $p^n$ elementos en $K$. 

2. La prueba de la unicidad nos dará pista para demostrar la existencia. Demostremos la unicidad. Sea $K$ un cuerpo con $p^n$. Consideremos el grupo de los unidades $U = K \setminus \{0\}$. Este grupo tiene $p^n - 1$ elementos y por el teorema de Lagrange, tenemos que $\forall \alpha \in U. \alpha^{p^n - 1} = 1$. Por tanto, los elementos de este grupo son raíces del polinomio $X^{p^n-1}-1$. Por tanto, los elementos de $K$ son las raíces del polinomio $f = X^{p^n}-X$ que podemos ver como un polinomio en $K[X]$. La unicidad se sigue de la unicidad del cuerpo de descomposición para un polinomio.

Para ver la existencia, demostramos que el cuerpo de descomposición para $f$ tiene exactamente $p^n$ elementos. Como $f' = -1$ tenemos que $f,f'$ no tienen raíces comunes y por tanto, no hay raíces múltiples de $f$, en consecuencia el cuerpo de descomposición contiene exactamente $p^n$ raíces (esto es lo mismo que decir que el polinomio es separable). Para terminar la prueba, necesito ver que el conjunto de dichas raíces es un subcuerpo del cuerpo de descomposición. En efecto, tenemos las siguientes propiedades, sean $u,v$ raíces de $f$ entonces:

\begin{enumerate}
\item $(u+v)^{p^n} - (u+v) = u^{p^n}-u+v^{p^n}-v = 0$
\item $(uv)^{p^n}-uv = u^{p^n}v^{p^n} - uv = 0$
\item $(-u)^{p^n} - (-u) = 0$
\item $(u^{-1})^{p^n} - u^{-1} = 0$
\item $(1)^{p_n} - 1 = 0$
\end{enumerate}
\end{proof}

\begin{definition}[Cuerpos finitos o de Galois]
El cuerpo de Galois con $q$ elementos es el único cuerpo con $q = p^n$ elementos, con $p$ un número primo. Lo notaremos por $\mathbb{F}_q$. 
\end{definition}

\begin{proposition}[Subgrupo multiplicativo finito de un cuerpo es cíclico]
	Sea $F$ un cuerpo arbitrario y $G$ un subgrupo finito de $F^x = F \setminus \{0\}$. Entonces, $G$ es cíclico.
	
	En particular, si $F$ es finito, su grupo de unidades es cíclico. 
\end{proposition}
\begin{proof}
	Sea $G$ un subgrupo finito de $F$. 
	
	Ya que $G$ es un grupo finito que será abeliano pues estamos asumiendo que los cuerpos tienen producto conmutativo. Por la clasificación de los grupos abelianos finitos mediante factores invariantes, tenemos que $G \cong \prod_{i = 1}^r C_{n_i}$ con $\forall 1 \le i < r. n_i|n_{i+1}$.
	
	Observemos que $mcm(n_i) = n_r$ y como para cada $a_i \in C_{n_i}$ se verifica que $a_i^{n_i} = 1$, también se verificará que $a_i^{n_r} = 1$. Por tanto, cualquier $a \in G$ verificará que $a^{n_r} = 1$. Por tanto, $a$ será raíz de $X^{n_r}-1$, que como mucho tiene $n_r$ raíces, esto me dice que $|G| \le n_r$. Pero por el isomorfismo tendría que ser $|G| = \prod_{i = 1}^r n_i = n_r$, de donde necesariamente todos los $n_i = 1$ salvo el último y tenemos que $G \cong C_{n_r}$. 
\end{proof}

\begin{theorem}[Estudio de la extensión $\mathbb{F}_q/\mathbb{F}_p$]
Sea $p$ un número primo y $q = p^n$ tenemos que:

\begin{enumerate}
\item $\mathbb{F}_q/\mathbb{F}_p$ es una extensión de Galois. 
\item $Gal(\mathbb{F}_q/\mathbb{F}_p)$ es un grupo cíclico de orden $n$ generado por el automorfismo de Frobenius. 
\end{enumerate}
\end{theorem}
\begin{proof}
\begin{enumerate}
\item Como por el teorema de Moore, $\mathbb{F}_q$ es el cuerpo de descomposición del polinomio $X^q - X$, la extensión es normal y finita. 

Como $\mathbb{F}_p$ es cuerpo finito, es perfecto. Como toda extensión finita de un cuerpo perfecto es separable, resulta que $\mathbb{F}_q/\mathbb{F}_p$ es separable. 

En consecuencia, la extensión es de Galois. 

\item Comencemos observando que por el teorema de Artin, $n = [\mathbb{F}_q:\mathbb{F}_p] = |Gal(\mathbb{F}_q/\mathbb{F}_p)|$. 

Como $\mathbb{F}_q$ es un cuerpo finito, es perfecto. Como $Car(\mathbb{F}_q) = p$ necesariamente, el homomorfismo de Frobenius $\phi$ es sobreyectivo. Y siempre es monomorfismo puesto que sale de un cuerpo. Por tanto, $\phi$ es automorfismo.  Además, fija $\mathbb{F}_p$ ya que $\phi(a) = a^p = a$, pues por el teorema de Moore $\mathbb{F}_p$ es el conjunto de las raíces de $X^p - X$. Por tanto, $\phi \in Gal(\mathbb{F}_q/\mathbb{F}_p)$.

Para $0 \le i < n$, $\phi^i \in Gal(\mathbb{F}_q/\mathbb{F}_p)$ y $\alpha \in \mathbb{F}_q$, tendríamos por inducción: $$\phi^i(\alpha) = \phi \phi^{i-1}(\alpha) = \phi(\alpha^{p^{i-1}}) = \phi(\alpha^{p^{i-1}}) = \alpha^{p^i}$$ Pero los $\phi^i$ son distintos ya que si $\phi^i = \phi^j$ con $i > j$ entonces $\phi^{i-j} = 1$ y entonces: $$\alpha = 1(\alpha) = \phi^{i-j}(\alpha) = \alpha^{p^{i-j}}$$ y entonces todo elemento de $\mathbb{F}_q$ sería raíz del polinomio $X^{p^{i-j}} - X$ que tiene un máximo de $p^{i-j} < p^n$ raíces. Esto daría un subcuerpo intermedio que sería cuerpo de descomposición de $X^{p^n}-X$ en contradicción de la definición de cuerpo de descomposición. 
\end{enumerate}
\end{proof}

\begin{proposition}[Retículo de subcuerpos de un cuerpo finito]
	Si denotamos por $Sub(F)$ a los subcuerpos de $F$ tendremos:
	
	$Sub(\mathbb{F}_{p^n}) = \{\mathbb{F}_{p^m}:m|n\}$
\end{proposition}
\begin{proof}
	Veamos la igualdad de conjuntos por doble inclusión. 
	
	$\subseteq)$ Supongamos que $\mathbb{F}_{q^m}$ es isomorfo a un subcuerpo de $\mathbb{F}_{p^n}$ entonces la aplicación inclusión de $\mathbb{F}_{q^m}$ en $\mathbb{F}_{p^n}$ debe ser un homomorfismo. Por las propiedades de la característica $p|q$ y como ambos son primos necesariamente $p = q$. 
	
	Por el teorema del grado como tenemos la torre $\mathbb{F}_{p} \subseteq \mathbb{F}_{p^m} \subseteq \mathbb{F}_{p^n}$ se tendrá que $$n = [\mathbb{F}_{p^n}:\mathbb{F}_{p}] = [\mathbb{F}_{p^n}:\mathbb{F}_{p^m}]m$$ y por tanto $m|n$. 
	
	$\supseteq)$ Supongamos que $m|n$ con $n = md$. Por lo anterior, $Gal(\mathbb{F}_{p^n}/\mathbb{F}_p)$ es cíclico de orden $n$ y generado por el automorfismo de Frobenius $\phi$. 
	
	Recordando el retículo de subgrupos de un grupo cíclico tiene que existir un único subgrupo $H$ de orden $d$ que será cíclico  y por tanto estará generado por $\phi^m$. 
	
	Utilización la conexión de Galois $H$ se corresponde con el subcuerpo $E = (\mathbb{F}_{p^n})^H$. Observemos que como $H$ es subgrupo de un cíclico, que es abeliano, $H$ debe ser normal y esto implica qque $E/\mathbb{F}_p$ es de Galois y $Gal(E/\mathbb{F}_p) \cong Gal(\mathbb{F}_{p^n}/\mathbb{F}_p)/Gal(\mathbb{F}_{p^n}/E)$ que tiene orden $n/d = m$ de donde $E = \mathbb{F}_{p^m}$ y por tanto, tenemos la implicación y la consecuencia del enunciado. 
\end{proof}

\begin{corollary}[Estudio de las extensiones finitas intermedias]
 Sea $p$ un número primo y $n,m \in \mathbb{N}$ con $m|n$, tenemos que:
 
 \begin{enumerate}
 \item $\mathbb{F}_{p^m} \subseteq \mathbb{F}_{p^n}$ es una extensión de Galois. 
 \item $Gal(\mathbb{F}_{p^n} /\mathbb{F}_{p^m}) \cong \langle \phi^m \rangle$
 \end{enumerate}
\end{corollary}
\begin{proof}
Es consecuencia de la demostración de la inclusión a la derecha del teorema anterior. 
\end{proof}

\subsection{Polinomios irreducibles sobre cuerpos finitos}

\subsubsection{Motivación}

Como motivación al estudio de los polinomios irreducibles sobre cuerpos finitos veamos cómo se puede construir cualquier cuerpo finito a partir de ellos:

\begin{proposition}[Construcción de cuerpos finitos con polinomios irreducibles]
Sea $m \in \mathbb{F}_p[X]$ un polinomio irreducible entonces $\mathbb{F}_{p^n} \cong \frac{\mathbb{F_p}[X]}{\langle m \rangle}$. 
\end{proposition}
\begin{proof}
Como $\mathbb{F}_p[X]$ es un DIP, los ideales maximales deben ser los elementos generados por irreducibles. Como $m$ es irreducibles, $\langle m \rangle$ es maximal y por tanto, $\frac{\mathbb{F_p}[X]}{\langle m \rangle}$ es un cuerpo. 

Para ver el número de elementos del cuerpo, observamos que cada clase $[a] \in \frac{\mathbb{F_p}[X]}{\langle m \rangle}$ tiene un único representante $r$ tal que $[r] = [a]$ y $gr(r) < gr(m)$. Esto lo da la unicidad del resto en el algoritmo de la división. Por otro lado, es claro que cualquier polinomio $r$ con $gr(r) < gr(m)$ nos da una clase del cociente. La unicidad garantiza que ambos conjuntos son biyectivos y por tanto, el cuerpo contiene $p^n$ elementos. 

Por el teorema de Moore se tiene que salvo isomorfismo hay un sólo cuerpo con $p^n$ elementos y de aquí se concluye el enunciado. 
\end{proof}

\subsubsection{Cálculo del número de polinomios irreducibles}

Vamos a empezar determinando el número de polinomios irreducibles de grado fijo que hay de grado $m$. Claramente, este número es finito para cada $m$. Además, podemos limitarnos a calcular el número de representates de las clases irreducibles para la relación de asociados. Posteriormente, si quisieramos calcular todos los irreducibles bastaría multiplicar el resultado por el número de unidades. Como hay $|K| - 1$ unidades (se quita el cero) en $K[X]$ bastaría multiplicar por $|K| - 1$. 

Sea $N_m = \{f \in \mathbb{F}_p[X]:\text{f es mónico irreducible de grado m}\}$

\begin{example}[Cálculo de $|N_1|$]
En $\mathbb{F}_p[X]$ los polinomios mónicos lineales son de la forma $x-a$ con $a \in \mathbb{F}_p[X]$ y son todos irreducibles. Por tanto, $|N_1| = p$.	
\end{example}

\begin{theorem}[Polinomios irreducibles de $\mathbb{F}_p$]
	Los factores irreducibles de $X^{p^n} - X$ en $\mathbb{F}_p[X]$ son exactamente los polinomios irreducibles de $\mathbb{F}_p[X]$ con grado divisor de $n$. En particular, $$X^{p^n} - X = \prod_{f \in N_m, m|n} f$$ y por tanto, $\sum_{m|n} m |N_m| = p^n$.
\end{theorem} 
\begin{proof}
$\Rightarrow$ Si $g$ es un factor irreducible de $X^{p^n} - X$ entonces vemos claramente que su grado divide a $n$. En efecto, $g$ tendrá alguna raíz $\alpha$ y $\alpha$ también es raíz de $X^{p^n} - X$ pues $g$ es factor de él. Por tanto, tenemos la siguiente torre de cuerpos $\mathbb{F}_{p^n} \supseteq \mathbb{F}_{p}(\alpha) \supseteq \mathbb{F}_p$. Por el teorema del grado, $gr(g) = [\mathbb{F}(\alpha):\mathbb{F}_p] | [\mathbb{F}_{p^n}:\mathbb{F}_p]$.
	
$\Leftarrow$ Los irreducibles de grado divisor de $n$ son factores. En efecto, sea $g$ un irreducible de grado $m$ con $m|n$. Si tomo una raíz $\alpha$ de $g$ en algún cuerpo de descomposición, observo que $\mathbb{F}_p(\alpha) \cong \mathbb{F}_{p^m}$ y como $m|n$ tenemos que $\mathbb{F}_{p^m} \subseteq \mathbb{F}_{p^n}$, esto es, las raíces están contenidas en las raíces de $X^{p^n} - X$ y entonces claramente $g|f$. 
	
Para obtener la descomposición en irreducibles, razonamos del siguiente modo. Por lo anterior, los factores son irreducibles con grado divisor. Cada factor puede aparecer únicamente una vez ya que si apareciera dos veces entonces teniendo en cuenta que, por el teorema de Moore, $X^{p^n}-X$ tiene $p^n$ raíces distintas obtendríamos menos raíces de las necesarias puesto que no habría una por cada unidad del grado. Además, como $X^{p^n}-X$ es mónico podemos tomar los factores también como polinomios mónicos. 
	
En resumen, podemos escribir $$X^{p^n} - X = \prod_{f \in N_m, m|n} f$$ Tomando grados en ambos miembros obtenemos la fórmula del enunciado. 
\end{proof}

\begin{corollary}[Cálculo con $n$ primo]
	Si $n$ es un número primo entonces $|N_n| = \frac{p^n-p}{n}$
\end{corollary}
\begin{proof}
	Obsérvese que $\sum_{m|n} m |N_m| = p^n$ y como $n$ es primo tenemos que $|N_1| + n|N_n| = p^n$. Pero $|N_1| = p$ y se sigue la igualdad del enunciado. 
\end{proof}

Combinamos lo anterior para $n$ no primo:

\begin{example}[Cálculo de $|N_4|$]
Por el corolario, $|N_2| = \frac{p^2 - p}{2}$ y por el teorema: $$p^4 = |N_4| = |N_1| + 2|N_2| + 4 |N_4| = p + 2 \frac{p^2 - p}{2} + 4|N_4| \implies |N_4| = \frac{p^4-p^2}{4}$$
\end{example}

Vamos a generalizar esta fórmulas para cualquier grado:

\begin{definition}[Función de Mobius]
	\[
	\mu(n) = 
	\begin{cases} 
	1 & n = 1 \\
	(-1)^s & n = \prod_{i = 1}^s p_i  \text{ para primos distintos } p_i \\
	0   & \text{en otro caso}
	\end{cases}
	\]
\end{definition}


\begin{theorem}[Fórmula de Gauss]
Sea $p$ un número primo y $N_n = \{f \in \mathbb{F}_p[X]:\text{f es mónico irreducible de grado n}\}$. Se tiene que:

$|N_n| = \frac{1}{n} \sum_{m|n} \mu(m) p^{\frac{n}{m}}$
\end{theorem}
\begin{proof}
La siguiente prueba se encuentra en \cite{counting-irreducible}. 
	
Se procede por inducción sobre $n$. 
	
Si $n = 1$ entonces el polinomio $x-a$ es irreducible en $\mathbb{F}_p$ para cualquier $a$ por tanto, $|N_1| = p = \mu(1)p$. 
	
Si $n > 1$ y definimos $R_n$ como el conjunto de las raíces de los polinomios de $N_n$. Podemos observar que dichas raíces pertenecen a $\mathbb{F}_{p^n}$ gracias al teorema que da los polinomios irreducibles sobre $\mathbb{F}_p$. Como las raíces en este cuerpo son todas distintas por el teorema de Moore, se deduce que $|R_n| = n |N_n|$. El problema se reduce a calcular este $|R_n|$. Pongamos $n = \prod_{i = 1}^k p_i^{e_i}$. Observamos que: $$R_n = \{\alpha \in \mathbb{F}_{p^n}:[\mathbb{F}_p(\alpha):\mathbb{F}_p] = n \} = \{ \alpha \in \mathbb{F}_{p^n}: \alpha \text{ no    contenido en ningún subcuerpo maximal de } \mathbb{F}_{p^n} \} = \mathbb{F}_{p^n} \setminus \cup_i \mathbb{F}_{p^{n/p_i}}$$ Finalmente, se utiliza el principio de inclusión-exclusión obteniéndose: $$|R_n| = p^n - | \cup_i \mathbb{F}_{p^{n/ p_i} }| = p^n - (\sum_{i = 1}^k p^{n/p_i}) + (\sum_{i,j = 1, i < j}^k p^{n/p_i p_j}) - \ldots +(-1)^k p^{n/ \prod p_i}$$ Recordando la expresión de la función de Mobius: $$\mu(1)p^{n/1}+\sum_{p_i |n} \mu(p_i)p^{n/p_i}+ \ldots \mu(\prod p_i) p^{n/\prod p_i}$$ Obsérvese que para cualquier otro divisor $\mu(d) = 0$. 
\end{proof}

Si se quieren los irreducibles no necesariamente mónicos bastaría multiplicar por $p-1$. 	

\begin{example}[Cálculo de $|N_6|$]
$$|N_6| = \frac{1}{6} \sum_{m|6} \mu(m) p^{\frac{6}{m}} =  \frac{1}{6} \Big(\mu(1)p^{6} + \mu(2)p^{3} + \mu(3)p^{2} + \mu(6)p\Big) = \frac{1}{6}(p^{6}-p^{2}-p^{3}+p)$$
\end{example}

La fórmula anterior nos confirma que existen polinomios irreducibles de cualquier grado sobre $\mathbb{F}_p$ cosa que no ocurre por ejemplo en $\mathbb{R}$. Recordemos que todo polinomio no constante con coeficientes en $\mathbb{R}$ admite una raíz en $\mathbb{C}$. Se observa que un polinomio con coeficientes en $\mathbb{R}$ que admita una raíz compleja, también admite su conjugada. Pero entonces, si $\chi$ es tal raíz, $(x-\chi)(x-\overline{\chi}) = x^2-(2Re(\chi))x + N(\chi)$ sería un factor con coeficientes reales del polinomio. Esto es, los únicos polinomios irreducibles sobre $\mathbb{R}$ son de grado 1 o 2. 

\begin{corollary}[Los cuerpos finitos tienen irreducibles de cualquier grado]
$\forall n \ge 1. |N_n| > 0$
\end{corollary}
\begin{proof}
$$N_n \ge \frac{1}{n} \sum_{m|n} \mu(\frac{n}{d}) p^d \ge \frac{1}{n} \Big(p^n - \sum_{i = 0}^{m-1} p^i \Big) = \frac{1}{n} \Big(p^n - \frac{p^n - 1}{p - 1} \Big)$$

Pero dado que $p \ge 2$, imponer que el último término sea mayor que cero equivale a que: $$p^{n+1} \ge 2p^n > 2p^n - 1$$ que es siempre cierto. 
\end{proof}

\subsubsection{Algoritmo de Berlekamp}

Dado un polinomio $f \in \mathbb{F}_p[X]$ queremos dar un algoritmo para determinar si es irreducible. 

Sea $f \in \mathbb{F}_p[X]$ un polinomio de grado $n > 1$ (los de grado 1 son todos irreducibles). Si el polinomio tiene raíces múltiples en un cuerpo de descomposición, no puede ser irreducible ya que, $\mathbb{F}_p$ es un cuerpo perfecto, y por tanto, los irreducibles son separables. En consecuencia, asumamos que $f$ no tiene raíces múltiples en un cuerpo de descomposición.

Sea $R = \frac{\mathbb{F}_p[X]}{\langle f \rangle}$ el cuerpo que define $f$. Ya vimos que todo elemento de $R$ se puede escribir como $\Big(\sum_{i = 0}^{n-1} a_iX^i \Big) + \langle f \rangle$ con $a_i \in \mathbb{F}_p$. Podemos ver $R$ como un espacio vectorial de dimensión $n$ sobre $\mathbb{F}_p$. En estas condiciones tenemos el análogo al endomorfismo de Frobenius, $T:R \to R$ tal que $T(g + \langle f \rangle) = g^p + \langle f \rangle$. Este endormofismo:

\begin{itemize}
\item Está bien definido. Pues si $g + \langle f \rangle = h + \langle f \rangle$ entonces $g = h + fB$ para $B \in \mathbb{F}_p[X]$ y por tanto por el binomio de Newton: $$g^p = (h+fB)^p = h^p + f^pB^p = h^p + f f^{p-1}B^p$$ de modo que $g^p + \langle f \rangle = h^p + \langle f \rangle$. 

\item Es lineal. Es decir $T(\alpha u + \beta v) = \alpha T(u) + \beta T(v)$ donde $u,v$ son los vectores de coeficientes de los polinomios que representan. De nuevo el truco consiste en aplicar la estrategia del apartado anterior. Si $u = (u_1, \ldots, u_n)$ y $v = (v_1,\ldots,v_n)$ entonces $T(u) = (u_1^p, \ldots, u_n^p)$ y $T(v) = (v_1^p,\ldots,v_n^p)$ de donde se sigue fácilmente el resultado. 
\end{itemize}

Claramente la identidad $1_R:R \to R$ también es lineal. 

\begin{theorem}[Principio del algoritmo de Berlekamp]
Sea $f \in \mathbb{F}_p[X]$ un polinomio de grado $n > 1$, y  $R = \frac{\mathbb{F}_p[X]}{\langle f \rangle}$. Entonces, $f$ es irreducible si y sólo la aplicación lineal $T - 1_R: R \to R$ tiene rango $n-1$.  
\end{theorem}
\begin{proof}
$\Rightarrow)$ Si $f$ es irreducible entonces $R$ es un cuerpo y $T$ sería el endomorfismo de Frobenius del cuerpo. Por tanto: $$Ker(T) = \{\alpha \in R: \alpha^p = \alpha \} = \mathbb{F}_p \implies dim(Ker(T)) = 1 \implies rg(T) = dim(Img(T)) = n-1$$

$\Leftarrow)$ Por contrarrecíproco, si $f$ es reducible entonces $f = gh$ con $g,h \in \mathbb{F}_p[X]$ con $gr(g),gr(h) \le gr(f)$. Además, como $f$ no tiene raíces múltiples, $g,h$ en el cuerpo de descomposición factorizan en lineales que no tienen factores comunes y por tanto, son primos relativos. Como el máximo común divisor no depende del cuerpo extensión, deducimos que $f,g$ son primos relativos en $\mathbb{F}_p$. Por el teorema de Bézout, $\exists A,B \in \mathbb{F}_p[X].Ag+Bh = 1$. Vamos a ver que: $$Ag,Bh \in Ker(T - 1_R) \iff (Ag)^p - Ag,(Bh)^p - Bh \in \langle f \rangle$$ Usando el teorema del binomio y que $f = gh$, tenemos que $$(Ag)^p = Ag(1-Bh)^{p-1} = Ag(1-(p-1)Bh+ \ldots + (-1)^{p-1}(Bh)^{p-1}) =  $$ $$ = Ag - gh(p-1)AB+\ldots+gh(-1)^{p-1}AB^{p-1}h^{p-2} \equiv Ag \; mod(f)$$ como queríamos demostrar. Finalmente, probamos que $Ag + \langle f \rangle, Bh + \langle f \rangle$ son linealmente independientes en $R$ con lo que $Ker(T - 1_R) \ge 2$ o equivalentemente $rg(T - 1_R) \le n-2$. Veamos que cualquier combinación igualada a cero es trivial: $$\exists a,b \in \mathbb{F}_p,C \in \mathbb{F}_p[X].aAg + bBh = ghC$$ pero como $mcd(g,h) = 1$, se deduce que $g|bB$ y $h|aA$. Pero al ser $Ag + Bh = 1 \implies mcd(g,B) = gcd(h,A) = 1$ y por el lema de Euclides, $g|b \land h|a$ por grados, necesariamente, $a = b = 0$
\end{proof}

Veamos ahora un ejemplo de aplicación del teorema anterior:

\begin{example}[Aplicación del principio de Berlekamp]
Dado el polinomio $f = X^5+X^4+1 \in \mathbb{F}_2[X]$. Determinar si es irreducible. 

\begin{enumerate}
\item El polinomio no tiene raíces múltiples pues $gcd(f,f') = gcd(X^5+X^4+1, 5X^4) = 1$. 

\item $R = \frac{\mathbb{F}_2[X]}{\langle f \rangle}$ es un espacio vectorial sobre $\mathbb{F}_2$ de dimensión 5, con base $1,x,x^2,x^3,x^4 + \langle f \rangle$.

\item Calculamos la expresión de $T$ a partir de las imágenes de los vectores de la base $T(1) = 1,T(x) = x^2,T(x^2) ) = x^4,T(x^3) = x^6 = 1+x+x^4,T(x^4) = x^8 = 1+x+x^2+x^3+x^4$. 

\item La matriz de $T - 1_R$ es:

\[
T - 1_R =
  \begin{bmatrix}
    1 & 0 & 0 & 1 & 1 \\
    0 & 0 & 0 & 1 & 1 \\
    0 & 1 & 0 & 0 & 1 \\
    0 & 0 & 0 & 0 & 1 \\
    0 & 0 & 1 & 1 & 1 
  \end{bmatrix}
  - \begin{bmatrix}
      1 & 0 & 0 & 0 & 0 \\
      0 & 1 & 0 & 0 & 0 \\
      0 & 0 & 1 & 0 & 0 \\
      0 & 0 & 0 & 1 & 0 \\
      0 & 0 & 0 & 0 & 1 
    \end{bmatrix} 
    = 
    \begin{bmatrix}
        0 & 0 & 0 & 1 & 1 \\
        0 & 1 & 0 & 1 & 1 \\
        0 & 1 & 1 & 0 & 1 \\
        0 & 0 & 0 & 1 & 1 \\
        0 & 0 & 1 & 1 & 0 
      \end{bmatrix}
\] 

\item Esta matriz tiene rango como mucho tres ya que la primera columna es cero y las tres últimas suman cero (estamos en característica 2). Como $3 < 4 = gr(f) - 1$ deducimos que $f$ es reducible. 
\end{enumerate}
\end{example}





