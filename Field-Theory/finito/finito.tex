\subsection{Resultados generales}

\begin{theorem}[Caracterización de los cuerpos finitos]
1. Si K es un cuerpo finito, entonces su $|K| = p^n$ con $p$ un número primo que es la característica del cuerpo y $n \ge 1$. 

2. Teorema de Moore: para cada primo $p$ y cada $n \ge 1$ existe un único cuerpo de orden $p^n$, este es, el cuerpo de descomposición del polinomio $X^{p^n} - X$ sobre $\mathbb{F}_p$. 
\end{theorem}
\begin{proof}
Si $K$ es finito, necesariamente su característica ha de ser un número primo y por tanto por el primer teorema de isomorfía $Img(f) \cong \mathbb{Z}_p$ es un subcuerpo de $K$. Subcuerpo que denotaremos por $\mathbb{F}_p$.

Ahora, $K$ tiene estructura de espacio vectorial sobre $\mathbb{F}_p$ y dado que $K$ es finito, considerando todos los elementos de $K$ como vectores del espacio vectorial sobre $\mathbb{F}_p$, claramente, $K$ es finitamente generado por sus elementos. Sabemos del álgebra lineal que todo espacio vectorial finitamente generado tiene una base. El cardinal de esta base nos dará la dimensión $n$ y todo vector de $K$ se expresa de forma en función de $n$ vectores. Por tanto, hay exactamente $p^n$ elementos en $K$. 

2. La prueba de la unicidad nos dará pista para demostrar la existencia. Demostremos la unicidad. Sea $K$ un cuerpo con $p^n$. Consideremos el grupo de los unidades $U = K-\{0\}$. Este grupo tiene $p^n - 1$ elementos y por el teorema de Lagrange, tenemos que $\forall \alpha \in U. \alpha^{p^n - 1} = 1$. Por tanto, los elementos de este grupo son raíces del polinomio $X^{p^n-1}-1$. Por tanto, los elementos de $K$ son las raíces del polinomio $f = X^{p^n}-X$ que podemos ver como un polinomio en $K[X]$. La unicidad se sigue de la unicidad del cuerpo de descomposición para un polinomio.

Para ver la existencia, demostramos que el cuerpo de descomposición para $f$ tiene exactamente $p^n$ elementos. Como $f' = -1$ tenemos que $f,f'$ no tienen raíces comunes y por tanto, no hay raíces múltiples de $f$, en consecuencia el cuerpo de descomposición contiene exactamente $p^n$ raíces (esto es lo mismo que decir que el polinomio es separable). Para terminar la prueba, necesito ver que el conjunto de dichas raíces es un subcuerpo del cuerpo de descomposición. En efecto, tenemos las siguientes propiedades, sean $u,v$ raíces de $f$ entonces:

\begin{enumerate}
\item $(u+v)^{p^n} - (u+v) = u^{p^n}-u+v^{p^n}-v = 0$
\item $(uv)^{p^n}-uv = u^{p^n}v^{p^n} - uv = 0$
\item $(-u)^{p^n} - (-u) = 0$
\item $(u^{-1})^{p^n} - u^{-1} = 0$
\item $(1)^{p_n} - 1 = 0$
\end{enumerate}
\end{proof}

\begin{definition}[Cuerpos de Galois]
El cuerpo de Galois con $q$ elementos es el único cuerpo con $q = p^n$ elementos, con $p$ un número primo. Lo notaremos por $\mathbb{F}_q$. 
\end{definition}

\begin{proposition}[Grupo multiplicativo de un cuerpo es cíclico]
	Todo subgrupo finito del grupo multiplicativo $F^x = F \setminus \{0\}$ de un cuerpo (no necesariamente finito $F$) es cíclico. 
\end{proposition}
\begin{proof}
	Sea $G$ un subgrupo finito de $F$. 
	
	Ya que $G$ es un grupo finito que será abeliano pues estamos asumiendo que los cuerpos tienen producto conmutativo. Por la clasificación de los grupos abelianos finitos mediante factores invariantes, tenemos que $G \cong \prod_{i = 1}^r C_{n_i}$ con $\forall 1 \ge i < r. n_i|n_{i+1}$.
	
	Observemos que $mcm(n_i) = n_r$ y como para cada $a_i \in C_{n_i}$ se verifica que $a_i^{n_i} = 1$, también se verificará que $a_i^{n_r} = 1$. Por tanto, cualquier $a \in G$ verificará que $a^{n_r} = 1$. Por tanto, $a$ será raíz de $X^{n_r}-1$, que como mucho tiene $n_r$ raíces, esto me dice que $|G| \le n_r$. Pero por el isomorfismo tendría que ser $|G| = \prod_{i = 1}^r n_i$, de donde necesariamente todos los $n_i = 1$ salvo el último y tenemos que $G \cong C_{n_r}$. 
\end{proof}

\begin{theorem}[Estudio de la extensión $\mathbb{F}_q/\mathbb{F}_p$]
Sea $p$ un número primo y $q = p^n$ tenemos que:

\begin{enumerate}
\item $\mathbb{F}_q/\mathbb{F}_p$ es una extensión de Galois. 
\item $Gal(\mathbb{F}_q/\mathbb{F}_p)$ es un grupo cíclico de orden $n$ generado por el automorfismo de Frobenius. 
\end{enumerate}
\end{theorem}
\begin{proof}
\begin{enumerate}
\item Como $\mathbb{F}_p$ es cuerpo finito es perfecto y resulta que la extensión $\mathbb{F}_q/\mathbb{F}_p$ es separable. Por el teorema de Moore, $\mathbb{F}_q$ es el cuerpo de descomposición del polinomio $X^q - X$ y por tanto, la extensión es normal. Esto es, la extensión es de Galois. 

\item Comencemos observando que por el teorema de Artin, $n = [\mathbb{F}_q:\mathbb{F}_p] = |Gal(\mathbb{F}_q/\mathbb{F}_p)|$. Además, el automorfismo de Frobenius $\phi \in Gal(\mathbb{F}_q/\mathbb{F}_p)$ ya que como $\mathbb{F}_q$ es un cuerpo finito y por tanto perfecto. Como es de característica prima, necesariamente $\phi$ es automorfismo. Y además, fija $\mathbb{F}_p$ ya que $\phi(a) = a^p = a$, por ejemplo porque $\mathbb{F}_p$ es cuerpo de descomposición de $X^p - X$. 

Para $0 \le i < n$, $\phi^i \in Gal(\mathbb{F}_q/\mathbb{F}_p)$. Tendríamos: $$\phi^i(\alpha) = \phi \phi^{i-1}(\alpha) = \phi(\alpha{p^{i-1}}) = \phi(\alpha{p^{i-1}}) = \alpha{p^i}$$ Finalmente, se comprueba que los $\phi^i$ son distintos ya que si $\phi^i = \phi^j$ con $i > j$ entonces $\phi{i-j} = 1$ y entonces $\alpha = 1(\alpha) = \phi^i(\alpha) = \alpha^{p^i}$ y entonces todo elemento de $\mathbb{F}_q$ sería raíz del polinomio $X^{p^i} - X$ que tiene un máximo de $p^i$ raíces. Contradicción. 
\end{enumerate}
\end{proof}

\begin{proposition}[Retículo de subcuerpos de un cuerpo finito y estudio de extensiones intermedias]
	Si denotamos por $Sub(F)$ a los subcuerpos de $F$ tendremos:
	
	$Sub(\mathbb{F}_{p^n}) = \{\mathbb{F}_{p^m}:m|n\}$
	
	En particular, la extensión $\mathbb{F}_{p^m} \subseteq \mathbb{F}_{p^n}$ es de Galois con grupo de Galois cíclico de orden $n/m$ y generado por $\phi^m$.
\end{proposition}
\begin{proof}
	Veamos la igualdad de conjuntos por doble inclusión. 
	
	$\subseteq)$ Supongamos que $\mathbb{F}_{q^m}$ es isomorfo a un subcuerpo de $\mathbb{F}_{p^n}$ entonces la aplicación inclusión de $\mathbb{F}_{q^m}$ en $\mathbb{F}_{p^n}$ debe ser un homomorfismo. Por las propiedades de la característica $p|q$ y como ambos son primos necesariamente $p = q$. 
	
	Por el teorema del grado como tenemos la torre $\mathbb{F}_{p} \subseteq \mathbb{F}_{p^m} \subseteq \mathbb{F}_{p^n}$ se tendrá que $$n = [\mathbb{F}_{p^n}:\mathbb{F}_{p}] = [\mathbb{F}_{p^n}:\mathbb{F}_{p^m}]m$$ y por tanto $m|n$. 
	
	$\supseteq)$ Supongamos que $m|n$ con $n = md$. Por lo anterior, $Gal(\mathbb{F}_{p^n}/\mathbb{F}_p)$ es cíclico de orden $n$ y generado por el automorfismo de Frobenius $\phi$. 
	
	Recordando el retículo de subgrupos de un grupo cíclico tiene que existir un único subgrupo $H$ de orden $d$ que será cíclico  y por tanto estará generado por $\phi^m$. 
	
	Utilización la conexión de Galois $H$ se corresponde con el subcuerpo $E = (\mathbb{F}_{p^n})^H$. Observemos que como $H$ es subgrupo de un cíclico, que es abeliano, $H$ debe ser normal y esto implica qque $E/\mathbb{F}_p$ es de Galois y $Gal(E/\mathbb{F}_p) \cong Gal(\mathbb{F}_{p^n}/\mathbb{F}_p)/Gal(\mathbb{F}_{p^n}/E)$ que tiene orden $n/d = m$ de donde $E = \mathbb{F}_{p^m}$ y por tanto, tenemos la implicación y la consecuencia del enunciado. 
\end{proof}


\subsection{Polinomios irreducibles sobre cuerpos finitos}

La representación de cuerpos finitos $\mathbb{F}_{p^n}$ como cocientes $\frac{F_p[X]}{f}$ con $f \in \mathbb{F}_p[X]$ un polinomio irreducible de grado $n$ es esencial en los computadores. Necesitamos por tanto, un buen conocimiento de los polinomios irreducibles sobre $\mathbb{F}_p[X]$.

\begin{proposition}[Raíces de un irreducible en $\mathbb{F}_p$]
	Sea $p \in \mathbb{F}_p[X]$ un irreducible de grado $d$. Consideremos un cuerpo extensión $F$ donde $p$ tiene una raíz $\alpha$. Entonces $p$ admite $d$ raíces distintas $\alpha^{p^i}$ con $i = 0,\cdots,d-1$. 
\end{proposition}
\begin{proof}
	\cite{link3}
\end{proof}

\begin{proposition}[Polinomios irreducibles de $\mathbb{F}_p$]
	Los factores irreducibles de $X^{p^n} - X$ en $\mathbb{F}_p[X]$ son exactamente los polinomios irreducibles de $\mathbb{F}_p[X]$ con grado divisor de $n$. 
\end{proposition} 
\begin{proof}
	$\Rightarrow$ Si $g$ es un factor irreducible de $X^{p^n} - X$ entonces vemos claramente que su grado divide a $n$. En efecto, $g$ tendrá alguna raíz $\alpha$ y $\alpha$ también es raíz de $X^{p^n} - X$ pues $g$ es factor de él. Por tanto, tenemos la siguiente torre de cuerpos $\mathbb{F}_{p^n} \supseteq \mathbb{F}_{p}(\alpha) \supseteq \mathbb{F}_p$. Por el teorema del grado, $gr(g) = [\mathbb{F}(\alpha):\mathbb{F}_p] | [\mathbb{F}_{p^n}:\mathbb{F}_p]$l
	
	$\Leftarrow$ Los irreducibles de grado divisor de $n$ son factores. En efecto, sea $g$ un irreducible de grado $m$ con $m|n$. Si tomo una raíz $\alpha$ de $g$ en algún cuerpo de descomposición, observo que $\mathbb{F}(\alpha) \cong \mathbb{F}_{p^m}$ y como $m|n$ tenemos que $\mathbb{F}_{p^m} \subseteq \mathbb{F}_{p^n}$, esto es, las raíces están contenidas en las raíces de $X^{p^n} - X$ y entonces claramente $g|f$. 
\end{proof}

Para más detalles véase \cite{link3}

Sea $N_m = \{f \in \mathbb{F}_p[X]:\text{f es mónico irreducible de grado m}\}$

\begin{proposition}
	$\sum_{m|n} m |N_m| = p^n$
\end{proposition}
\begin{proof}
	Vimos en la prueba del teorema de Moore que $X^{p^n} - X$ es separable y en particular sus raíces en cuerpo de descomposición son todas simples. Esto implica que sus factores irreducibles en $\mathbb{F}_p[X]$ son todos distintos. Además como es mónico, podemos obtener su factorización como producto de polinomios mónicos. 
	
	La proposición anterior nos dice que debemos considerar como factores todos los irreducibles sobre $\mathbb{F}_p[X]$ que tengan grado divisor de $n$. En resumen, podemos escribir $$X^{p^n} - X = \prod_{f \in N_m, m|n} f$$ Tomando grados en ambos miembros obtenemos la fórmula del enunciado. 
\end{proof}

\begin{corollary}
	Si $n$ es un número primo entonces $|N_n| = \frac{p^n-p}{n}$
\end{corollary}
\begin{proof}
	Obsérvese que $\sum_{m|n} m |N_m| = p^n$ y como $n$ es primo tenemos que $|N_1| + n|N_n| = p^n$. Pero $|N_1| = p$ y se sigue la igualdad del enunciado. 
\end{proof}

Pero no sólo podemos calcular los polinomios mónicos irreducibles de grado primo. Veamos un ejemplo:

\begin{example}
	En $\mathbb{F}_p[X]$ los polinomios mónicos lineales son de la forma $x-a$ con $a \in \mathbb{F}_p[X]$ y son todos irreducibles. Por tanto, $N_1 = p$.
	
	El teorema implica que $p^2 = 2 |N_2| + |N_1| = 2 |N_2| + p$ luego $|N_2| = \frac{p^2 - p}{2}$, análogamente, $|N_4| = \frac{p^4 - p^2}{4}$.
\end{example}

Vamos a generalizar esta fórmulas para el cálculo del número de polinomios irreducibles. 

\begin{definition}[Función de Mobius]
	\[
	\mu(n) = 
	\begin{cases} 
	1 & n = 1 \\
	(-1)^s & n = \prod_{i = 1}^s p_i  \text{ para primos distintos } p_i \\
	0   & \text{en otro caso}
	\end{cases}
	\]
\end{definition}


\begin{theorem}
	$N_n = \frac{1}{n} \sum_{m|n} \mu(m) p^{\frac{n}{m}}$
\end{theorem}
\begin{proof}
	Ver \cite{counting-irreducible}
\end{proof}




