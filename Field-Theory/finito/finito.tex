\subsection{Existencia y unicidad}

\begin{proposition}
Sea $A$ un anillo. Entonces existe un único homomorfismo $f:\mathbb{Z} \to A$ tal que $f(1) = 1 \land f(n) = n \cdot 1$.
\end{proposition}
\begin{proof}
Defino $f: \mathbb{Z} \to A$ como $n \to n \cdot 1$. $f$ está bien definido ya que si $n \in \mathbb{N}$ entonces $n \cdot 1$ está de forma natural definiado y si $n \in \mathbb{Z}$ entonces $n1 = (-n)(-1)$ también está naturalmente definido. 

Por otro lado, $f(1) = 1 \cdot 1 = 1$ y es un homomorfismo ya que $$f(n+m) = (n+m) \cdot 1 = n \cdot 1 + m \cdot 1 = f(n) + f(m)$$

Claramente cualquier otro homomorfismo verificando las condiciones sería igual a $f$. 
\end{proof}

\begin{definition}
Sea $A$ un anillo.

La característica de $A$, $car(A)$, es el número entero positivo o nulo $n$ tal que $Ker(f) = n\mathbb{Z}$. 
\end{definition}

Obsérvese que es una buena definición ya que $\mathbb{Z}$ es un dominio de ideales principales.

\begin{proposition}[Propiedades de la característica]
1. Si $Car(A) = 0$ entonces $f(\mathbb{Z}) \cong \mathbb{Z}$. Si $n = Car(A) \neq 0$ entonces $f(\mathbb{Z}) \cong \mathbb{Z}_n$ y además $\forall a \in A.na  = 0$. 

2. Si $A$ es un dominio de integridad entonces $Car(A)$ es cero o un número primo. 

3. Sea $A$ un anillo de característica prima $p$ entonces la aplicación $R(a) = a^p$ es un endomorfismo conocido como endomorfismo de Frobenius.

4. Si $f: A \to B$ es un homomorfismo entonces $car(B)|car(A)$. 
\end{proposition}
\begin{proof}
1. Utilícese el primer teorema de isomorfía y la definición de característica.

2. Si $Car(A) = 0$ hemos acabado. Si $n = Car(A) \neq 0$ entonces si $n$ no fuera primo, $0 = n \cdot 1 = (n_1n_2) \cdot 1 = (n_1 \cdot 1)(n_2 \cdot 1) \implies n_1 \cdot 1 = 0 \lor n_2 \cdot 1 = 0$ donde hemos utilizado la propiedad asociativa general.  

3. Escribamos $(\alpha+\beta)^p = \alpha^p + \beta^p + \sum \binom{p}{i} \alpha{p-i}\beta^i$ con $1 \le i \le p-1$. Claramente, $p$ divide a cada uno de los coeficientes binomiales y como $F$ tiene característica $p$, se deduce que $(\alpha+\beta)^p = \alpha^p+\beta^p$. Las otras propiedades son triviales.

4. $0 = f(0) = f(1 Car(A)) = 1 Car(A)$ y como $Car(B)$ debe ser el mínimo que anule a uno, tendrá que ser $Car(B)|Car(A)$.  
\end{proof}

\begin{theorem}
1. Si K es un cuerpo finito, entonces su $|K| = p^n$ con $p$ un número primo que es la característica del cuerpo y $n \ge 1$. 

2. Teorema de Moore: para cada primo $p$ y cada $n \ge 1$ existe un único cuerpo de orden $p^n$, este es, el cuerpo de descomposición del polinomio $X^{p^n} - X$ sobre $\mathbb{F}_p$. 
\end{theorem}
\begin{proof}
Si $K$ es finito, necesariamente su característica ha de ser un número primo y por tanto por el primer teorema de isomorfía $Img(f) \cong \mathbb{Z}_p$ es un subcuerpo de $K$. Subcuerpo que denotaremos por $\mathbb{F}_p$.

Ahora, $K$ tiene estructura de espacio vectorial sobre $\mathbb{F}_p$ y dado que $K$ es finito, considerando todos los elementos de $K$ como vectores del espacio vectorial sobre $\mathbb{F}_p$, claramente, $K$ es finitamente generado por sus elementos. Sabemos del álgebra lineal que todo espacio vectorial finitamente generado tiene una base. El cardinal de esta base nos dará la dimensión $n$ y todo vector de $K$ se expresa de forma en función de $n$ vectores. Por tanto, hay exactamente $p^n$ elementos en $K$. 

2. La prueba de la unicidad nos dará pista para demostrar la existencia. Demostremos la unicidad. Sea $K$ un cuerpo con $p^n$. Consideremos el grupo de los unidades $U = K-\{0\}$. Este grupo tiene $p^n - 1$ elementos y por el teorema de Lagrange, tenemos que $\forall \alpha \in U. \alpha^{p^n - 1} = 1$. Por tanto, los elementos de este grupo son raíces del polinomio $X^{p^n-1}-1$. Por tanto, los elementos de $K$ son las raíces del polinomio $f = X^{p^n}-X$ que podemos ver como un polinomio en $K[X]$. La unicidad se sigue de la unicidad del cuerpo de descomposición para un polinomio.

Para ver la existencia, demostramos que el cuerpo de descomposición para $f$ tiene exactamente $p^n$ elementos. Como $f' = -1$ tenemos que $f,f'$ no tienen raíces comunes y por tanto, no hay raíces múltiples de $f$, en consecuencia el cuerpo de descomposición contiene exactamente $p^n$ raíces (esto es lo mismo que decir que el polinomio es separable). Para terminar la prueba, necesito ver que el conjunto de dichas raíces es un subcuerpo del cuerpo de descomposición. En efecto, tenemos las siguientes propiedades, sean $u,v$ raíces de $f$ entonces:

\begin{itemize}
\item $(u+v)^{p^n} - (u+v) = u^{p^n}-u+v^{p^n}-v = 0$
\item $(uv)^{p^n}-uv = u^{p^n}v^{p^n} - uv = 0$
\item $(-u)^{p^n} - (-u) = 0$
\item $(u^{-1})^{p^n} - u^{-1} = 0$
\item $(1)^{p_n} - 1 = 0$
\end{itemize}
\end{proof}

\begin{proposition}
	Sea $F^x = F - \{0\}$ el grupo de las unidades para el producto del cuerpo finito $F$. Se verifica que $F^x$ es un grupo cíclico de orden $|F|-1$. 
\end{proposition}
\begin{proof}
	Ver \cite{link2}
\end{proof}

\begin{proposition}
	Si $f \in \mathbb{F}_p[X]$ entonces el número de raíces de $f$ in $\mathbb{F}_{p^n}$ es el grado del polinomio $gcd(f,X^{p^n}-X)$.
\end{proposition}
\begin{proof}
	Ver Cox, 291.
\end{proof}

\begin{proposition}[Retículo de subcuerpos de un cuerpo finito]
	Sean $\mathbb{F}_{q^m}, \mathbb{F}_{p^n}$ cuerpos finitos. Entonces $\mathbb{F}_{q^m}$ es isomorfo a un subcuerpo de $\mathbb{F}_{p^n}$ si y sólo si $p = q \land m|n$. 
\end{proposition}
\begin{proof}
	$\Rightarrow)$ Supongamos que $\mathbb{F}_{q^m}$ es isomorfo a un subcuerpo de $\mathbb{F}_{p^n}$ entonces la aplicación inclusión de $\mathbb{F}_{q^m}$ en $\mathbb{F}_{p^n}$ debe ser un homomorfismo. Por las propiedades de la característica $p|q$ y como ambos son primos necesariamente $p = q$. 
	
	Por el teorema del grado como tenemos la torre $\mathbb{F}_{p} \subseteq \mathbb{F}_{p^m} \subseteq \mathbb{F}_{p^n}$ se tendrá que $$n = [\mathbb{F}_{p^n}:\mathbb{F}_{p}] = [\mathbb{F}_{p^n}:\mathbb{F}_{p^m}]m$$ y por tanto $m|n$. 
	
	$\Leftarrow)$ Hay que utilizar extensiones de Galois o bien utilizar otras herramientas de la característica como en \cite{4}.
\end{proof}


\subsection{Polinomios irreducibles sobre cuerpos finitos}

La representación de cuerpos finitos $\mathbb{F}_{p^n}$ como cocientes $\frac{F_p[X]}{f}$ con $f \in \mathbb{F}_p[X]$ un polinomio irreducible de grado $n$ es esencial en los computadores. Necesitamos por tanto, un buen conocimiento de los polinomios irreducibles sobre $\mathbb{F}_p[X]$.

\begin{proposition}[Raíces de un irreducible en $\mathbb{F}_p$]
	Sea $p \in \mathbb{F}_p[X]$ un irreducible de grado $d$. Consideremos un cuerpo extensión $F$ donde $p$ tiene una raíz $\alpha$. Entonces $p$ admite $d$ raíces distintas $\alpha^{p^i}$ con $i = 0,\cdots,d-1$. 
\end{proposition}
\begin{proof}
	\cite{link3}
\end{proof}

\begin{proposition}
	Los factores irreducibles de $X^{p^n} - X$ en $\mathbb{F}_p[X]$ son exactamente los polinomios irreducibles de $\mathbb{F}_p[X]$ con grado divisor de $n$. 
\end{proposition} 
\begin{proof}
	$\Rightarrow$ Si $g$ es un factor irreducible de $X^{p^n} - X$ entonces vemos claramente que su grado divide a $n$. En efecto, $g$ tendrá alguna raíz $\alpha$ y $\alpha$ también es raíz de $X^{p^n} - X$ pues $g$ es factor de él. Por tanto, tenemos la siguiente torre de cuerpos $\mathbb{F}_{p^n} \supseteq \mathbb{F}_{p}(\alpha) \supseteq \mathbb{F}_p$. Por el teorema del grado, $gr(g) = [\mathbb{F}(\alpha):\mathbb{F}_p] | [\mathbb{F}_{p^n}:\mathbb{F}_p]$l
	
	$\Leftarrow$ Los irreducibles de grado divisor de $n$ son factores. En efecto, sea $g$ un irreducible de grado $m$ con $m|n$. Si tomo una raíz $\alpha$ de $g$ en algún cuerpo de descomposición, observo que $\mathbb{F}(\alpha) \cong \mathbb{F}_{p^m}$ y como $m|n$ tenemos que $\mathbb{F}_{p^m} \subseteq \mathbb{F}_{p^n}$, esto es, las raíces están contenidas en las raíces de $X^{p^n} - X$ y entonces claramente $g|f$. 
\end{proof}

Para más detalles véase \cite{link3}

Sea $N_m = \{f \in \mathbb{F}_p[X]:\text{f es mónico irreducible de grado m}\}$

\begin{proposition}
	$\sum_{m|n} m |N_m| = p^n$
\end{proposition}
\begin{proof}
	Vimos en la prueba del teorema de Moore que $X^{p^n} - X$ es separable y en particular sus raíces en cuerpo de descomposición son todas simples. Esto implica que sus factores irreducibles en $\mathbb{F}_p[X]$ son todos distintos. Además como es mónico, podemos obtener su factorización como producto de polinomios mónicos. 
	
	La proposición anterior nos dice que debemos considerar como factores todos los irreducibles sobre $\mathbb{F}_p[X]$ que tengan grado divisor de $n$. En resumen, podemos escribir $$X^{p^n} - X = \prod_{f \in N_m, m|n} f$$ Tomando grados en ambos miembros obtenemos la fórmula del enunciado. 
\end{proof}

\begin{corollary}
	Si $n$ es un número primo entonces $|N_n| = \frac{p^n-p}{n}$
\end{corollary}
\begin{proof}
	Obsérvese que $\sum_{m|n} m |N_m| = p^n$ y como $n$ es primo tenemos que $|N_1| + n|N_n| = p^n$. Pero $|N_1| = p$ y se sigue la igualdad del enunciado. 
\end{proof}







