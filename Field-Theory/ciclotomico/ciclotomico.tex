\subsection{Raíces de la unidad}

\begin{definition}[Raíces n-ésimas de la unidad]
Dado un cuerpo $K$ y una clausura algebraica $\overline{K}$. 

Las raíces n-ésimas de la unidad son las raíces del polinomio $X^n-1$ en $\overline{K}$.
\end{definition}

\begin{proposition}[Condición para que las raíces n-ésimas de la unidad sean distintas]
Sea $K$ un cuerpo. 

Las raíces n-ésimas de la unidad son todas distintas si y sólo si $Car(K) \nmid n$. En particular, si la característica es cero. 

Si $0 \neq p = Car(K) | n$ entonces si $n = mp^r$ existen $m$ raíces n-ésimas de la unidad distintas y todas tienen multiplicidad $p^r$. 
\end{proposition}
\begin{proof}
Las raíces serán distintas si y sólo si $X^n - 1$ no tiene raíces mútiples. Por el criterio de las raíces simples, esto ocurre cuando $mcd(f,Df) = 1$. En nuestro caso, cuando $mcd(X^n - 1,nX^{n-1}) = 1$. Distinguimos tres casos. Si $0 \neq p = Car(K) | n$ entonces claramente $mcd(X^n - 1,nX^{n-1}) = mcd(X^n - 1,0) = X^n - 1 \neq 1$. Si $Car(K) = 0 \lor p = Car(K) \nmid n$ entonces el polinomio $nX^{n-1}$ tiene grado $n-1$ y el único irreducible que lo divide es $X$ como $X$ no divide a $X^n - 1$ se deduce que ambos son primos relativos. 

Si $0 \neq p = Car(K) | n$ entonces el criterio de la derivada nos dice que existen raíces múltiples. En particular, como $p \neq 0$ tenemos el endomorfismo de Frobenius, que nos dice que si $n = mp^r$ entonces $X^n - 1 = (X^m - 1)^{p^r}$ de modo que aplicando el caso anterior existen $m$ raíces n-ésimas de la unidad distintas y tienen multiplicidad $p^r$. 
\end{proof}

En lo sucesivo asumiremos que existen $n$ raíces n-ésimas distintas de la unidad. Suponemos también que el cuerpo $K$ es fijo. 

\begin{proposition}[Raíces n-ésimas como subgrupo cíclico]
Las raíces n-ésimas de la unidad forman un subgrupo cíclico de orden $n$ del grupo multiplicativo de $\overline{K}$ que denotamos $\mu_n$.
\end{proposition}
\begin{proof}
Para ver que es un subgrupo tomo $a,b$ raíces n-ésimas. Entonces $(ab^{-1})^n = a^nb^{-n} = 1$ de donde $ab^{-1}$ también es una raíz n-ésima de la unidad. Además, el subgrupo es cíclico pues todo subgrupo finito del grupo multiplicativo de un cuerpo es cíclico como ya vimos en cuerpos finitos. 
\end{proof}

\begin{definition}[Raíz n-ésimas primitiva de la unidad]
Una raíz n-ésima primitiva de la unidad es un generador del grupo de las raíces n-ésimas de la unidad. 
\end{definition}

\begin{proposition}[Retículo de raíces n-ésimas]
$Sub(\mu_n) = \{\mu_d:d | n \}$. Además, si $a$ es una raíz n-ésima primitiva, esto es, $\mu_n = \langle a \rangle$ entonces $a^{n/d}$ es una raíz d-ésima primitiva de la unidad, esto es, $\mu_d = \langle a^{n/d} \rangle$. 
\end{proposition}
\begin{proof}
Esta proposición es la clasificación de los grupos cíclicos de orden finito, que se tiene de la teoría de grupos. 
\end{proof}

\subsection{Extensiones ciclotómicas}

\begin{definition}[Extensión ciclotómica]
Una extensión ciclotómica de $K$ es un cuerpo de descomposición sobre $K$ de un polinomio del tipo $X^n-1$. Equivalentemente, es una extensión simple generada por una raíz n-ésima primitiva de la unidad. 
\end{definition}

\begin{theorem}[Estructura de Galois de una extensión ciclotómica]
Sea $F/K$ una extensión ciclotómica generada por el polinomio $X^n-1$ y sea $\xi$ raíz n-ésima primitiva de la unidad. Entonces:

\begin{enumerate}
\item $F/K$ es una extensión de Galois. 
\item $Gal(F/K) \cong H \le U(\mathbb{Z}_n)$ mediante el isomorfismo $\sigma \mapsto i$ tal que $\sigma(\xi) = \xi^i$.
\item $|Gal(F/K)| | \phi(n)$. 
\end{enumerate}
\end{theorem}
\begin{proof}
\begin{enumerate}
\item $F/K$ es una extensión normal por ser el cuerpo de descomposición del polinomio $X^n - 1$ sobre $K$. Hemos asumido además que las raíces de $X^n - 1$ son distintas. En particular, también lo son las raíces de sus factores irreducibles, esto es, el polinomio es separable. Como el cuerpo de descomposición está formado por sus raíces, necesariamente, la extensión es separable. Por tanto, es de Galois. 

\item  Tenemos que $\sigma \in Gal(F/K)$ queda determinado por su valor sobre $\xi$. Queremos ver que si $\sigma(\xi) = \xi^i$ entonces $i \in U(\mathbb{Z}_n)$. 

En efecto, si tomamos $j$ tal que $\sigma^{-1}(\xi) = \xi^j$ entonces $\xi = (\sigma \sigma^{-1})( \xi) = \xi^{ji}$. Como $\xi$ tiene orden $n$, esta ecuación se formula de manera equivalente como $ij = 1$ en $\mathbb{Z}_n$ y por tanto $i,j \in U(\mathbb{Z}_n)$ como queríamos. 

Por tanto, la aplicación dada en el enunciado está bien definida. Además es un homomorfismo de grupos ya que si $\tau \in Gal(F/K)$ viene dada por $\tau(\xi) = \xi^k$ entonces $\sigma \circ \tau (\xi) = \xi^{ik}$. 

Además, si $\sigma \in Ker$ entonces $\sigma(\xi) = \xi^i = \xi$ de modo que $\sigma = Id_F$ y por tanto, tenemos un monomorfismo. Por el primer teorema de isomorfía, $Gal(F/K) \cong Im \le U(\mathbb{Z}_n)$.

\item Es consecuencia del teorema de Lagrange y de que $\phi(n) = |U(\mathbb{Z}_n)|$. 
\end{enumerate}
\end{proof}

\subsection{Polinomios ciclotómicos}

\begin{definition}[Polinomio ciclotómico]
Sea $F/K$ una extensión ciclotómica. 

El n-ésimo polinomio ciclotómico en $K$ es $\Phi_n(X) = \prod (X-a_i)$ con $a_i$ las raíces n-ésimas primitivas de la unidad en $F$. 
\end{definition}

\begin{proposition}[Propiedades de los polinomios ciclotómicos]
Se verifican las siguientes propiedades:

\begin{enumerate}
\item $X^n - 1 = \prod_{d|n} \Phi_d(X)$
\item Los coeficientes de $\Phi_n(X)$ pertenecen al cuerpo característico de $K$. Si $Car(K) = 0$ entonces pertenecen a $\mathbb{Z}$. Además, $gr(\Phi_n(X)) = \phi(n)$ y $\Phi_n(X)$ es mónico. 
\item Cálculo del n-ésimo polinomio ciclotómico: $\Phi_n(X) = \frac{X^n - 1}{\prod_{d|n \land d \neq n} \Phi_d(X)}$ 
\item Fórmula de Gauss: 
$\Phi_n(x) = \prod_{d\mid n} (x^{d}-1)^{\mu(n/d)}$
\end{enumerate}
\end{proposition}
\begin{proof}
\begin{enumerate}
\item Escribimos $\phi_d(X) = \prod X-\xi_i$ con $\xi_i$ raíces d-ésimas de la unidad. Es claro, que todas estas raíces son raíces n-ésimas de la unidad cuyo orden es $d$ y recíprocamente. Es claro, que juntando estos productos obtenemos $X^n - 1$ como producto de los $\phi_d$ para cada $d|n$. 

\item Se hace por inducción sobre $n$.
  
\item Basta ver que hay $\phi(n)$ raíces primitivas de las unidad. Esto es así ya que un grupo cíclico de orden $n$ tiene $\phi(n)$ generadores distintos como se vio en la teoría de grupos. Por tanto, $\phi_n$ tiene grado $\phi(n)$. 

\item  
\end{enumerate}
\end{proof}

Una forma de recordar la fórmula de Gauss y la fórmula que relaciona distintos ciclótomicos es ver que hay una dualidad entre $\phi$ y polinomios $X^m - 1$. Además, en nuestras fórmulas, $\mu$ siempre a alternado un argumento simétrico tipo $d, \frac{m}{d}$. 


\subsection{Polinomios ciclotómicos con coeficientes racionales}

\cite{cox} ofrece una versión restringida a $\mathbb{Q}$ de los resultados generales en cualquier cuerpo a partir de la página 231. Puede ser de provecho. En particular, las raíces, tienen una expresión explícita en términos de la exponencial compleja. 

\begin{proposition}[Irreducibilidad de los polinomios ciclotómicos racionales]
Para cada $n \in \mathbb{N}$, $\Phi_n(X)$ es irreducible sobre $\mathbb{Q}$.
\end{proposition}
\begin{proof}
Sea $\xi$ una raíz n-ésima primitiva de la unidad y $f(X) = Irr(\xi,\mathbb{Q})$. Veamos que $\phi_n(X) = f(X)$ con lo que $\phi_n(X)$ será irreducible. 
\end{proof}

\begin{corollary}[Grado de las extensiones ciclótomicas]
Sea $\xi$ una raíz n-ésima primitiva de la unidad entonces:

$[\mathbb{Q}(\xi):\mathbb{Q}] = gr(\Phi_n(X)) = \phi(n)$
\end{corollary}


\begin{proposition}[Grupo de Galois de las extensiones ciclotómicas racionales]
Sea $\xi$ una raíz n-ésima primitiva de la unidad, entonces:

$Gal(\mathbb{Q}(\xi)/\mathbb{Q}) \cong U(\mathbb{Z}_n)$ mediante el isomorfismo $\sigma \mapsto l$ con $\sigma(\xi) = \xi^l$. 
\end{proposition}
\begin{proof}
Obsérvese que como $\phi_n$ es irreducible en $\mathbb{Q}$, necesariamente, $\sigma(\xi)$ será raíz de $\phi_n$. Como el grupo de las raíces n-ésimas de la unidad es cíclico, tendremos que, $\sigma(\xi) = \xi^l$ donde $l$ será primo relativo con $n$ (debe ser raíz primitiva y por tanto, el orden del subgrupo que genera debe ser $n$). Esto prueba que la aplicación $f$ que dará el isomorfismo está bien definida. 

En estas condiciones, $f$ es un monomorfismo pues si $\sigma(\xi) = l \land \eta(\xi) = k$ entonces $$(\sigma \circ \eta)(\xi) = \sigma(\eta(\xi)) = \xi^{kl}$$ y por tanto, $f(\sigma \circ \eta) = kl = f(\sigma) f(\eta)$. Por tanto, $f$ es un homomorfismo. Que es inyectivo se sigue de que: $$f(\sigma) = 1 \implies \sigma(\xi) = \xi \implies \sigma = Id$$

Como $|Gal(\mathbb{Q}(\xi)/\mathbb{Q})| = [\mathbb{Q}(\xi):\mathbb{Q}] = \phi(n) = |U(\mathbb{Z}_n)|$, tenemos una aplicación inyectiva entre conjuntos de igual cardinal finito, luego debe ser biyectiva. 
\end{proof}








