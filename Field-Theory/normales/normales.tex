\subsection{Extensiones conjugadas}

\begin{definition}[Elementos conjugados]
$u,v \in \overline{K}$ son conjugados si se verifica alguna de las siguientes condiciones equivalentes.
\end{definition}

\begin{proposition}[Caracterización de elementos conjugados]
Sean $u,v \in \overline{K}$. Entonces:

\begin{enumerate}
\item $Irr(u,K) = Irr(v,K)$. 
\item $\exists \sigma: K(u) \to K(v)$ isomorfismo sobre $K$ tal que $\sigma(u) = v$
\item $\exists \sigma:K(u) \to \overline{K}$ homomorfismo sobre $K$ tal que $\sigma(u) = v$
\item $\exists \sigma:\overline{K} \to \overline{K}$ automorfismo sobre $K$ tal que $\sigma(u)= v$
\end{enumerate}
\end{proposition}
\begin{proof}
Veamos circulamente las implicaciones.

\begin{enumerate}
\item $K(u) \cong \frac{K[X]}{\langle f(X) \rangle} \cong K(v)$. Este isomorfismo deja invariante a $K$ pues la evaluación de una constante es la constante. 
\item Basta tomar como codominio $\overline{K}$.
\item Como $\overline{K}$ es la clausura algebraica de $K$ por la transitividad de la clausura también es clausura algebraica de $K(u)$, en particular, la extensión $\frac{\overline{K}}{K(u)}$ es algebraica. En estas condiciones teníamos que se podía extender el homorfismo desde la extensión algebraica $\overline{K}$ hasta la clausura algebraica $\overline{K}$. Pero todo endomorfismo sobre extensiones algebraicas es automorfismo.
\item Trabajamos con la extensión del automorfismo $\sigma$ a anillos de polinomios $\overline{\sigma}$. Sea $f = Irr(u,K) = \sum a_iX^i$. Entonces $$0 = \overline{\sigma}(f(u)) = \sum \sigma(a_i)\sigma(u)^i = \sum a_i v^i$$ como este isomorfismo lleva irreducibles en irreducibles pues también $f = Irr(v,K)$. 
\end{enumerate}
\end{proof}

\begin{definition}[Extensiones conjugadas]
Dos extensiones algebraicas $\frac{F_1}{K},\frac{F_2}{K}$ son conjugadas si ocurre cualquiera de las condiciones de la siguiente proposición. 
\end{definition}

\begin{proposition}
Dadas dos extensiones algebraicas $\frac{F_1}{K},\frac{F_2}{K}$ son equivalentes:

\begin{enumerate}
\item Existe un isomorfismo $\sigma:F_1 \to F_2$ sobre $K$. 
\item Existe un homomorfismo $\sigma:F_1 \to \overline{K}$sobre $K$ tal que $\sigma(F_1) = F_2$.
\item Existe un isomorfismo $\sigma:\overline{K} \to \overline{K}$ sobre $K$ tal que $\sigma(F_1) = F_2$. 
\end{enumerate}
\end{proposition}

\subsection{Extensiones normales}

\begin{definition}[Extensión normal]
Una extensión normal $\frac{F}{K}$ es un subcuerpo de la clausura de $K$ que determina una extensión algebraica y tal que se verifica alguna de las siguientes condiciones.
\end{definition}

\begin{proposition}[Condiciones equivalentes para extensión normal]
Dada una extensión algebraica $\frac{F}{K}$ tal que $F$ un subcuerpo de la clausura de $K$. Son equivalentes:

\begin{enumerate}
\item Para todo homomorfismo $\sigma:F \to \overline{K}$ sobre $K$ se verifica que $\sigma(F) = F$. 
\item Todo polinomio irreducible sobre $K$ que tiene una raíz en $F$ descompone en factores lineales en $F[X]$. 
\item $F$ es el cuerpo de descomposición de una familia de polinomios. 
\end{enumerate}
\end{proposition}
\begin{proof}
Veamos las implicaciones de forma circular:

\begin{enumerate}
\item Sea $f$ un irreducible con $u$ una raíz en $F$. Si $f$ no es lineal entonces tendrá alguna raíz $v$ en un cuerpo extensión. Pero las raíces del mismo irreducibles son elementos conjugados sobre $K$ y por tanto existe un automorfismo en $\overline{K}$ que lleva $u$ en $v$. Este automorfismo restringido a $F$ es un homomorfismo y como todo homomorfismo de $F$ a la clausura deja fijo a $F$, en particular, $v \in F$. Se procede por inducción para demostrar que $F$ descompone en lineales sobre $F[X]$. 
\item Dada $u \in F$, por hipótesis, $Irr(u,F)$ tiene todas sus raíces en $F$. Claramente, $F$ será el cuerpo de descomposición de los polinomios mínimos de sus elementos.
\item Sea $S$ las raíces de la familia de polinomios que tiene a $F$ como su cuerpo de descomposición. Entonces $F = K(S)$. 

Sea $\sigma:F \to \overline{K}$ un homomorfismo sobre $K$. $\sigma(F) \subseteq F$ ya que $\sigma(K) = K$ y si tomo una raíz $u$ un polinomio sobre $K$ entonces $\sigma(u)$ es raíz de ese mismo polinomio. Por tanto, $\sigma$ se puede ver como un endomorfismo en $F$. Como las extensionesson algebraicas todo endomorfismo es automorfismo y por tanto $\sigma(F) = F$. 
\end{enumerate}
\end{proof}

\begin{proposition}[Extensiones normales finitas]
Sea $\frac{E}{K}$ una extensión de cuerpos. 

$E$ es el cuerpo de descomposición de un polinomio $p \in K[X]$ $\iff$ $\frac{E}{K}$ es normal y finita.
\end{proposition}

\begin{example}
	Veamos que $\mathbb{Q}(\sqrt[3]{2})$ no es el cuerpo de descomposición de ningún
	polinomio de $\mathbb{Q}[X]$.
	
	Tengo que el polinomio $x^3-2$ es irreducible sobre $\mathbb{Q}[X]$ y tiene una raíz sobre $\mathbb{Q}(\sqrt[3]{2})$, entonces por la proposición anterior, forzaríamos a que $X^3-2$ descompusiera completamente sobre $\mathbb{Q}(\sqrt[3]{2})$. Pero esto no es posible ya que este cuerpo se queda en $\mathbb{Q}(\sqrt[3]{2}) \subseteq \mathbb{R}$ y en particular no contiene a las raíces complejas $\omega \sqrt[3]{2},\omega^2 \sqrt[3]{2}$.
\end{example}

\begin{proposition}[Propiedades de las extensiones normales]
Se verifican las siguientes propiedades:

\begin{enumerate}
\item Sea $\frac{E}{K}$ una extensión normal y $\frac{F}{K}$ una extensión algebraica. Entonces la extensión $\frac{EF}{F}$ es una extensión normal. 
\item Sea $K \subseteq F \subseteq E$ una torre de cuerpos con $\frac{E}{K}$ normal. Entonces $\frac{E}{F}$ es una extensión normal. 
\item Sean $\frac{F_1}{K},\frac{F_2}{K}$ dos extensiones normales. Entonces $\frac{F_1F_2}{K}$ es una extensión normal. 
\item Sea $\frac{F_\lambda}{K}$ con $\lambda \in \Lambda$ una familia de extensiones normales y sea $E = \cap_\lambda F_\lambda$ entonces la extensión $\frac{E}{K}$ es normal. 
\end{enumerate}
\end{proposition}
\begin{proof}
\begin{enumerate}
\item Si $\sigma:EF \to \overline{K}$ es un homomorfismo sobre $K$ entonces $\sigma(EF) = \sigma(E)\sigma(F) = EF$ y por tanto la extensión $\frac{EF}{K}$ es normal. 
\item Sea $\sigma:E \to \overline{F}$ un homomorfismo sobre $F$. Tenemos que observar que como $\frac{E}{K}$ es normal en particular asumimos que es una extensión algebraica. Esto nos dice que, en particular, $\frac{F}{K}$ es algebraica y gracias a esto podemos aplicar la transitividad de la clausura algebraica para deducir que $\overline{F} = \overline{K}$. Entonces $\sigma:E \to \overline{K}$ es un homomorfismo que en particular fija $K$. Por ser $\frac{E}{K}$ una extensión normal se tiene que verificar que $\sigma(E) = E$ y por tanto, $\frac{E}{F}$ es normal. 
\end{enumerate}
\end{proof}

\begin{theorem}[Extensión de homomorfismos a una extensión normal]
Sea $K \subseteq F \subseteq E$ una torre de cuerpos con $\frac{E}{K}$ normal. Entonces todo homomorfismo $\tau:F \to E$ se extiende a un automorfismo $\overline{\tau}:E \to E$. 
\begin{tikzcd}
E \arrow[dashed]{r}{\overline{\tau}} \arrow[dash]{d} &
E  & \\
F \arrow{ur}{\tau} \arrow[dash]{d} & \\
K 
\end{tikzcd}
\end{theorem}
\begin{proof}
Sea $\tau:F \to E$ un homomorfismo. Cambiamos el codominio por $\overline{K}$ y seguimos teniendo un homomorfismo. Pero, ya que $\frac{E}{K}$ es algebraica por ser normal, podemos extender el homomorfismo a $\tau:E \to K$. Finalmente, por la normalidad este homomorfismo verifica que $\tau(E) = E$. De modo que tenemos un endomorfismo $\tau:E \to E$ y todo homomorfismo de extensiones algebraicas es un automorfismo.
\end{proof}

\subsection{Clausura normal}

\begin{definition}[Clausura normal]
Sea $\frac{F}{K}$ una extensión algebraica, la clausura normal de $\frac{F}{K}$ es una extensión $\frac{E}{K}$ con $$E = \cap \{H:H \supseteq F \land \frac{H}{K} \text{ es normal}\}$$
\end{definition}

\begin{proposition}[Existencia y unicidad de la clausura normal]
Para toda extensión algebraica $\frac{F}{K}$existe una clausura normal $\frac{E}{K}$ además es única salvo isomorfismo. 
\end{proposition}
\begin{proof}
La existencia se sigue del de que la intersección que define a la clausura normal es no vacía ya que $\overline{K}$ siempre define una extensión normal utilizando que es algebraica y que todo endomorfismo será un automorfismo.

Si tengo dos clausuras normalescomo son intersección de normales y ambas lo son deben ser las misma!! (revisar)
\end{proof}

\begin{proposition}[Clausura de una extensión de generación finita]
Sea $F = K(u_1,\ldots,u_n)$ y $f_i = Irr(u_i,K)$.

La clausura normal de $F$ es el cuerpo de descomposición del polinomio $f = \prod f_i$. 
\end{proposition}
\begin{proof}
El cuerpo de descomposición de $f$ determina una extensión normal. Además, esta extensión es la mínima que puede ser normal. Continuará. 
\end{proof}

\begin{proposition}
La clausura normal de una extensión finita es finita. 
\end{proposition}

\subsection{Polinomio normal}

\begin{definition}
Un polinomio irreducible es normal si en todo extensión algebraica que tenga una raíz del polinomio, este descompone en factores lineales. 
\end{definition}

\begin{proposition}
Las siguientes propiedades son equivalentes:

\begin{enumerate}
\item f es normal sobre $K$.
\item El cuerop de descomposición de $f$ sobre $K$ es $K(u)$ con $u$ una raíz de $f$. 
\item Todas las raíces de $f$ se expresan como polinomios en una cualquiera de ellas. 
\end{enumerate}
\end{proposition}

