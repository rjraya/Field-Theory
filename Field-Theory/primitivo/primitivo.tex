\begin{definition}[Elemento primitivo y extensiones simples]
Sea $F/K$ una extensión de cuerpos.

$a \in F$ es primitivo para $F/K$ si $F = K(\alpha)$ en cuyo caso se dice que $F/K$ es una extensión simple. 
\end{definition}

\begin{theorem}[Teorema de Steinitz o del elemento primitivo]
Sea $F/K$ una extensión finita. 

$F/K$ es simple $\iff$ Existe un número finito de cuerpos intermedios. 
\end{theorem}

\begin{proposition}[Ejemplos de extensiones primitivas]
Los siguientes son ejemplos de extensiones simples:

\begin{enumerate}
\item Toda subextensión de una extensión de Galois finita.
\item Toda extensión separable finita.
\end{enumerate}
\end{proposition}
\begin{proof}
\begin{enumerate}
\item Sea $F/K$ subextensión de $E/K$ una extensión de Galois finita. Por ser $E/K$ finita, el grupo de Galois es finito. Y por tanto, existe un número finito de subgrupos que pueden dar cuerpos intermedios entre $F$ y $K$. 
\item Como la extensión es finita, en particular es algebraica y por tanto, existe una clausura normal, $E/K$. De exte modo, nuestra extensión $F/K$ es una subextensión de la extensión de Galois $E/K$ y por tanto sera simple. 
\end{enumerate}
\end{proof}

Una vez determinado cuando una extensión es simple y presentados algunos ejemplos, damos criterios en ciertas situaciones para reconocer a los elementos primitivos.

\begin{proposition}[Elementos primitivos en extensiones de Galois]
Sea $E/K$ una extensión de Galois y $\alpha \in E$.

$\alpha$ es primitivo $\iff$ los conjugados de $\alpha$ son distintos.
\end{proposition}
\begin{proof}
$\Rightarrow)$ Si $\alpha$ es primitivo, 
\end{proof}









