\subsection{Extensiones separables}

\begin{definition}[Extensión separable]
Un elemento algebraico $u$ sobre un cuerpo $K$ es separablesi $Irr(u,K)$ no tiene raíces múltiples.

Una extensión algebraica $\frac{F}{K}$ es separable si todos los elementos de $F$ son separables sobre $K$.  
\end{definition}

\begin{exercise}
Buscar una extensión normal que no sea separable.
\end{exercise}

\begin{proposition}[Torres separables]
Sea $E \supseteq F \supseteq K$ una torre de cuerpos. 

Si $\frac{E}{F}$ es una extensión separable entonces 
$\frac{E}{F},\frac{F}{K}$ son extensiones separables. 
\end{proposition}
\begin{proof}
$\frac{F}{K}$ es separable ya que todo elemento de $E$ es  sobre $K$ y todo elemento de $F$ está en $E$. 

Para ver que $\frac{E}{F}$ es separable sea $u \in E$ entonces claramente $Irr(u,F)|Irr(u,K)$ ya que $gr(Irr(u,F)) \le gr(Irr(u,K))$ y como $Irr(u,K)(u) = 0$ claramente $Irr(u,K) \in \langle Irr(u,F) \rangle$ y como $Irr(u,K)$ no tiene raíces múltiples, tampoco puede tenerlas $Irr(u,F)$. 
\end{proof}

\subsection{Cuerpos perfectos}
