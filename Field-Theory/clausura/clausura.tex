

\begin{lemma}
	Sea $K$ un cuerpo. Son equivalentes:
	
	1. Todo polinomio $f  \in K[X]$ no constante tiene al menos una raíz en $K$.\\
	2. Todo polinomio $f  \in K[X]$ no constante descompone en $K$. \\
	3. Los polinomios irreducibles en $K[X]$ son los polinomios de grado 1. \\
	4. $K$ no tiene extensiones algebraicas propias. \\
	5. No existen extensiones algebraicas $\frac{F}{K}$ con $F \neq K$. 
\end{lemma}


\begin{definition}[Cuerpo algebraicamente cerrado]
	Un cuerpo $K$ es algebraicamente cerrado si cumple cualquiera de las condiciones anteriores. 
\end{definition}

\begin{theorem}[Teorema de Steinitz]
	Si $K$ es un cuerpo existe un cuerpo algebraicamente cerrado $F$ extensión de $K$. 
\end{theorem}

\begin{corollary}
	Si $K$ es un cuerpo, existe una extensión de $K$ que es algebraica y algebraicamente cerrada. 	
\end{corollary}

\begin{definition}[Clausura algebraica]
	Una extensión de cuerpos $\frac{F}{K}$ es una clausura algebraica de $K$ si 
	
	1. Es una extensión algebraica. \\
	2. $F$ es un cuerpo algebraicamente cerrado. 
\end{definition}



